\documentclass[12pt]{article}
\usepackage[utf8]{inputenc}
\usepackage{amsthm}
\usepackage{url}
\usepackage[most]{tcolorbox}
\usepackage{colortbl}
\usepackage{philex}
\usepackage{booktabs}
\usepackage{tabu}
\usepackage{tikz-dependency}
\theoremstyle{definition}
\newtheorem{exmp}{Example}[section]
\usepackage[a4paper, top=2.5cm, bottom=2.5cm, left=1.5cm, right=1.5cm]{geometry}


% ↓ this package enables Japanese-text input anywhere without using commands (added by Yada 2022-05-20)
\usepackage[whole]{bxcjkjatype}
\usepackage{ulem}


% colors: https://latexcolor.com/
\definecolor{seabornOrange}{HTML}{DE8F05}
\definecolor{fuchsia}{HTML}{966FD6}
\definecolor{asparagus}{HTML}{006B3C}
\definecolor{ashgrey}{HTML}{B2BEB5}
\definecolor{babyblue}{HTML}{A1CAF1}
\definecolor{banana}{HTML}{FFE135}
\definecolor{bittersweet}{HTML}{FE6F5E}
\definecolor{brass}{HTML}{B5A642}
\definecolor{sienna}{HTML}{E75480}
\definecolor{burlywood}{HTML}{DEB887}
\definecolor{celadon}{HTML}{ACE1AF}
\definecolor{chestnut}{HTML}{CD5C5C}
\definecolor{cyan}{HTML}{008B8B}
\definecolor{dollarbill}{HTML}{85BB65}
\definecolor{heliotrope}{HTML}{DF73FF}
\definecolor{indianyellow}{HTML}{E3A857}
\definecolor{airforceblue}{HTML}{5D8AA8}

% The following are macros for ENTITY ANNOTATION
% generic entity
\newcommand{\entity}[1]{\colorbox{seabornOrange}{#1}}
% DRUG and different versions containing attributes
\newcommand{\drug}[1]{\colorbox{brass}{#1$_{\texttt{drug}}$}}
\newcommand{\drugStop}[1]{\colorbox{brass}{#1$_{\texttt{drug}}^{\textcolor{red}{stopped}}$}\ }
\newcommand{\drugStart}[1]{\colorbox{brass}{#1$_{\texttt{drug}}^{\textcolor{red}{started}}$}\ }
\newcommand{\drugInc}[1]{\colorbox{brass}{#1$_{\texttt{drug}}^{\textcolor{red}{increased}}$}\ }
\newcommand{\drugDec}[1]{\colorbox{brass}{#1$_{\texttt{drug}}^{\textcolor{red}{decreased}}$}\ }
\newcommand{\drugUD}[1]{\colorbox{brass}{#1$_{\texttt{drug}}^{\textcolor{red}{unique\_dose}}$}\ }
% ANATOMY
\newcommand{\anatomy}[1]{\colorbox{dollarbill}{#1$_{\texttt{anatomy}}$}\ }
% DISORDER & Negated DISORDER
\newcommand{\disorder}[1]{\colorbox{fuchsia}{\textcolor{white}{#1$_{\texttt{disorder}}$}}\ }
\newcommand{\disorderNeg}[1]{\colorbox{fuchsia}{\textcolor{white}{#1}$_{\textcolor{white}{\texttt{disorder}}}^{\textcolor{red}{negated}}$}\ }
% TEST & Negated Test
\newcommand{\test}[1]{\colorbox{asparagus}{\textcolor{white}{#1}$_{\textcolor{white}{\texttt{test}}}$}\ }
\newcommand{\testNeg}[1]{\colorbox{asparagus}{#1$_{\texttt{test}}^{\textcolor{red}{negated}}$}\ }
% FUNCTION
\newcommand{\function}[1]{\colorbox{banana}{#1$_{\texttt{function}}$}\ }
% EVALUATION & positive / negative / neutral EVALUATION
\newcommand{\eval}[1]{\colorbox{babyblue}{#1$_{\texttt{eval}}$}\ }
\newcommand{\evalPos}[1]{\colorbox{babyblue}{#1$_{\texttt{eval}}^{\textcolor{red}{positive}}$}\ }
\newcommand{\evalNeg}[1]{\colorbox{babyblue}{#1$_{\texttt{eval}}^{\textcolor{red}{negative}}$}\ }
\newcommand{\evalNeut}[1]{\colorbox{babyblue}{#1$_{\texttt{eval}}^{\textcolor{red}{neutral}}$}\ }
% MEASURE
\newcommand{\measure}[1]{\colorbox{bittersweet}{#1$_{\texttt{measure}}$}\ }
% TIME expressions and correspondign attributes
\newcommand{\timex}[2]{\colorbox{ashgrey}{#1$_{\texttt{time}}$}}
\newcommand{\timexDur}[2]{\colorbox{ashgrey}{#1$_{\texttt{time}}^{\textcolor{red}{duration}}$}\ }
\newcommand{\timexFreq}[2]{\colorbox{ashgrey}{#1$_{\texttt{time}}^{\textcolor{red}{frequency}}$}\ }
\newcommand{\timexPIT}[2]{\colorbox{ashgrey}{#1$_{\texttt{time}}^{\textcolor{red}{rel.~point~in~time}}$}\ }
\newcommand{\timexDate}[2]{\colorbox{ashgrey}{#1$_{\texttt{time}}^{\textcolor{red}{date}}$}\ }
% ROUTE
\newcommand{\route}[1]{\colorbox{sienna}{#1$_{\texttt{route}}$}\ }
% TRIGGER
\newcommand{\trigger}[1]{\colorbox{celadon}{#1$_{\texttt{change\_trigger}}$}\ }
% OTHER
\newcommand{\other}[1]{\colorbox{burlywood}{#1$_{\texttt{other}}$}\ }
% USER
\newcommand{\user}[1]{\colorbox{chestnut}{#1$_{\texttt{user}}$}\ }
% URL
\newcommand{\link}[1]{\colorbox{cyan}{#1$_{\texttt{url}}$}\ }
% PERSONAL INFORMATION
\newcommand{\info}[1]{\colorbox{heliotrope}{#1$_{\texttt{perso\_info}}$}\ }
% DOCTOR
\newcommand{\doctor}[1]{\colorbox{indianyellow}{#1$_{\texttt{doctor}}$}\ }

% two short-cuts to write frequent terms in texttt mode
\newcommand{\dis}{\texttt{disorder}\ }
\newcommand{\dr}{\texttt{drug}\ }
\newcommand{\cause}{$<$\texttt{cause}$>$\ }
\newcommand{\conseq}{$<$\texttt{consequence}$>$\ }
% The following are macros for RELATION ANNOTATION:
% #1: argument 1
% #2: argument 2 of the relation 
\newcommand{\causedRel}[2]{\depedge[edge unit distance=0.8ex, label style={fill=brass, font= \large}]{#1}{#2}{\texttt{caused}}}

\newcommand{\treatmentRel}[2]{\depedge[edge unit distance=0.5ex, label style={fill=brass, font= \large}]{#1}{#2}{\texttt{treatment\_for}}}

\newcommand{\dosageRel}[2]{\depedge[edge unit distance=0.5ex, label style={fill=brass, font= \large}]{#1}{#2}{\texttt{dosage}}}

\newcommand{\experiencedRel}[2]{\depedge[edge unit distance=0.5ex, label style={fill=airforceblue, font= \large}]{#1}{#2}{\texttt{experienced\_in}}}

\newcommand{\examinedRel}[2]{\depedge[edge unit distance=0.5ex, label style={fill=brass, font= \large}]{#1}{#2}{\texttt{examined\_with}}}

\newcommand{\refersRel}[2]{\depedge[edge unit distance=0.5ex, label style={fill=brass, font= \large}]{#1}{#2}{\texttt{refers\_to}}}

\newcommand{\resultedRel}[2]{\depedge[edge unit distance=0.5ex, label style={fill=brass, font= \large}]{#1}{#2}{\texttt{resulted\_in}}}

\newcommand{\interactedRel}[2]{\depedge[edge unit distance=0.5ex, label style={fill=brass, font= \large}]{#1}{#2}{\texttt{interacted\_with}}}

\newcommand{\changeRel}[2]{\depedge[edge unit distance=0.5ex, label style={fill=brass, font= \large}]{#1}{#2}{\texttt{change}}}

\newcommand{\timeRel}[2]{\depedge[edge unit distance=0.5ex, label style={fill=brass, font= \large}]{#1}{#2}{\texttt{time}}}

\newcommand{\discontRel}[2]{\depedge[edge unit distance=0.5ex, hide label, dashed]{#1}{#2}{\texttt{discontinuous}}}

% the RELATIONS environment; only the sentence chunks have to be defined in the respective texts
\newcommand{\drawRelations}[2]{
    \begin{dependency}[theme=default]
        \begin{deptext}
            #1 \\   % here goes the text, already split with "\&" into chunks
        \end{deptext}
        #2          % here go the relations (defined above), as many as you like, e.g. \causedRel{1}{5}, \causedRel{3}{8}
    \end{dependency}
}

% the EXAMPLE environment -- there is probably a much better solution.
\newcommand{\examples}[5]{%
    {\small
    \lb{#1}{
        \lba[en]{en}{#2}
        \lbb[fr]{fr}{#3}
        \lbb[de]{de}{#4}
        \lbz[ja]{ja}{#5}
    }}
}



\title{Annotation Guidelines\\ Version 07/22}

\author{KEEPHA}
\date{July 2022}

\usepackage[hidelinks]{hyperref}
\begin{document}

\maketitle

\tableofcontents
\clearpage

\section{Introduction}
In the following, we describe the annotation scheme we developed for our multi-lingual corpus KEEPHA.
The data for this corpus consists of different sources, e.g., patient forums, social media like Twitter,  \textcolor{red}{TBD}.
In Table \ref{tab:data}, we give an overview of the current annotated documents, their origin and the language they are written in.

\begin{table}[h]
\centering
\begin{tabular}{@{}llll@{}}
\toprule
\textbf{name}   & {\textbf{language}}   & \textbf{\#examples}   & \textbf{source} \\ \midrule

Lifeline-v1     &  German (de)          & 4,169                 &  www.lifeline.de \\ 

                & French (fr)           &                       &                   \\
lexapro\_ade    & Japanese (ja)         & 560                   & Twitter                  \\
chiebukuro & Japanese (ja) & 59 & Yahoo! JAPAN Q\&A\\
\midrule

Total           & de, fr, ja            &                       & \\

\bottomrule
\end{tabular}
\caption{Overview of the current KEEPHA dataset.}
\label{tab:data}
\end{table}


\noindent We first define the entities we annotated (Section \ref{sec:entities}), including their attributes (Section \ref{sec:attributes}), using examples to highlight their usage.
Subsequently, we define the relations connecting these entities (Section \ref{sec:relations}).
English examples are extracted from the CADEC corpus \textcolor{red}{CITATION}, while German examples are taken from the Lifeline data (v1 and unlabeled data).
The French examples are from AFA.
Lastly, the Japanese examples are taken from social media (Twitter) and forum (Yahoo! JAPAN Q\&A) posts.
\textbf{**} denotes completely made-up examples, while \textbf{*} marks examples that are translations from another language. 


\section{Entities}\label{sec:entities}

In Table \ref{tab:all_entities}, we show all available entities and their respective attributes.

\begin{table}[h]
\centering
% \renewcommand{\arraystretch}{1.3}
\small
\begin{tabular}{@{}lll@{}}
\toprule
\textbf{section} & \textbf{entity}   & {\textbf{attributes}} \\ \midrule

\ref{ent_drug} &\drug{\dr}                          & \texttt{increase}, \texttt{decrease}, \texttt{stopped}, \texttt{started}, \texttt{unique\_dose} \\
\ref{ent_disorder}&\disorder{\dis}                     & \texttt{negated}\\
\ref{ent_anatomy}&\anatomy{\texttt{anatomy}}          & \\              
\ref{ent_test}&\test{\texttt{test}}                & \\          
\ref{ent_function}&\function{\texttt{function}}        & \\                
\ref{ent_eval}&\eval{\texttt{evaluation}}    & \texttt{positive}, \texttt{negative}, \texttt{neutral} \\
\ref{ent_measure}&\measure{\texttt{measure}}          & \\
\ref{ent_timex}&\timex{\texttt{time}}~             & \texttt{frequency}, \texttt{duration}, \texttt{date}, \texttt{point} \\
\ref{ent_route}&\route{\texttt{route}}              & \\
\ref{ent_drug_status}&\trigger{\texttt{drug\ status change trigger}}  & \\
\ref{ent_user}&\user{\texttt{user}}                & \\
\ref{ent_url}&\link{\texttt{url}}                 & \\
\ref{ent_personal}&\info{\texttt{personal information}}   & \\
\ref{ent_doctor}&\doctor{\texttt{doctor}}            & \\
\ref{ent_other}&\other{\texttt{other}}              & \\

\bottomrule
\end{tabular}
\caption{Overview of the entities and their attributes.}
\label{tab:all_entities}
\end{table}



\subsection{General Guidelines for Entity Annotation}

We now define general rules for the annotation of entities.

\subsubsection*{Scope}

Regarding scope, we annotate entities in form of noun or verb phrases together with their modifying parts, e.g.~adjectives, adverbs etc., though complex post modifiers (e.g., relative clauses: ``a pain that is \ldots"; and PP modifiers: ``a neck pain from yesterday") are excluded.
We always prefer the smallest core noun phrase. 
If this does not work, a long span of the whole verb phrase (or even the whole clause/sentence) is allowed.

\examples{ex:scope}
    {I have \disorder{absolutely no appetite} and \disorder{constantly feel sick} to my \anatomy{stomach}.}
    {(\ldots) ça fait \disorder{drolement mal} et \disorder{très genant}. }
    {(\ldots) ich hatte in der Situation auch \disorder{absolut keinen Hunger}.}
    {\disorder{急性骨髄性白血病}と診断されました} % I was diagnosed with acute myeloid leukemia.
        % {全く食欲がない,胃がずっとムカムカする}

\subsubsection*{Metaphors and descriptive language}

Further, we annotate descriptive expressions based on the \textit{patient's perspective} even if they do not occur in a medical dictionary.
We always try to follow the patient's perspective: If the patient thinks something (e.g., a symptom, an irregularity) is a disorder, then we annotate it as \dis.

\examples{ex:descriptions}
    {(\ldots) I felt like I would imagine someone that had \disorder{MS} would feel like, \disorder{swaying}, \disorder{pulling myself up stairs}, (\ldots)}
    {(\ldots) J'ai \disorder{failli tomber dans les pommes}, (\ldots)}
    {Die \anatomy{Beine} \disorder{fühlen sich nicht stark} man hat \disorder{Gefühle des Ungleichgewichts}.}
    {*\drug{マイスリー}を飲むと\anatomy{脳}が\disorder{強制終了}する} % Zolpidem makes my brain shut down

%Seit zwei Tagen ist meine \entity{Übelkeit}$^{\attribute{negated}}$ wieder weg, doch da dieses Kopfkino mich vollkommen fertig macht und mir viel an Lebensqualität und -freude nimmt, .. || Ich habe nun das Problem, dass meine Haut sehr juckt. || Ich habe auch immer wieder mal seltsames Herzstolpern. Manchmal überschlägt sich das Herz, dann setzt es mal wieder kurz aus, manchmal hab ich für ein paar Sekunden ein Rasen, ||



Often, patients use metaphors to describe their suffering or, on the other side, a good experience with medication therapy.
Since metaphors are descriptive language as well, we annotate them as we would annotate explicit symptom descriptions.
Later, we can re-investigate those ``special disorders''.

\examples{ex:metaphors}
    {I felt always like I had a \disorder{veil over my head} and \disorder{lead in my legs}.}
    {(\ldots) à chaque effort, mon \anatomy{cœur} \disorder{bat la chamade} et jai ne n'ai  \disorder{pas beaucoup de souffle}.}
    {(\ldots) nur \timexPIT{1 Woche später}~war ich \evalPos{wie im 7. Himmel}.}
    {ずっと\anatomy{お腹}が\disorder{ゴロゴロ}いってる} % My stomach keeps saying grrrr



\subsubsection*{Determiners and Possessives}

We do not include determiners or possessive pronouns in an entity.

\examples{determiners}
    {*I only really felt the \disorder{burning sensation} in the \anatomy{arms} and \anatomy{chest} when I ...}
    {(\ldots) J'ai développé une \disorder{très forte sinusite chronique},(\ldots)}
    {\timexDur{Seit zwei Tagen}~ist meine \disorderNeg{Übelkeit} wieder weg, doch da ...}
    {(no determiners in Japanese)}


\subsubsection*{Enumerations}

Consecutive entities or enumerations that are, for example, separated by commas or conjunctions are to be split apart and annotated separately.

\examples{enumerations}
    {I felt \disorder{pain} in my \anatomy{arms}, \anatomy{legs}, \anatomy{ears} and \anatomy{toes}.}
    {\measure{4} \other{opèrations},  \drug{himurel}, \drug{pentasa}, \drug{methotrexate}, (\ldots)}
    % {*Ich fühle die \disorder{Schmerzen} in meinen \anatomy{Armen}, \anatomy{Beinen}, \anatomy{Ohren} und \anatomy{Zehen}.}
    {\timexPIT{Vor einem Jahr}~ bekam ich kurz hintereinander zwei \disorder{Herzinfarkte}, einen \disorder{Schlaganfall} und eine \disorder{Embolie}.}
    {\anatomy{まつ毛}や\anatomy{眉毛}も\disorder{抜けてしまった}ので} % Because my eyelashes and eyebrows have fallen out too.

\subsubsection*{Nested Entities}

Nested entities are not permitted; we always annotate the largest phrase.
For example, in the case of ``headache,'' we do not split the word into ``head'' and ``ache''. 
The general rule is to apply a syntax-based decision, i.e.~basic noun phrases (without preposition, particle, etc.) are annotated as one entity.

This is also language-dependant:
\begin{itemize}
    \item German: do not split compound words, e.g.~``\disorder{Kopfschmerzen}'' is annotated as one entity.
    \item Japanese: 
    \begin{itemize}
        \item do not split compound words based on standard tokenization
        \item split noun phrases (including compound nouns) at ``prepositions'' (particles て, に, を, は, の)
    \end{itemize}
    
    \item French: basic noun phrase (``\disorder{douleur thoracique}'') is annotated as one \dis, otherwise  we separate the annotations, e.g.~``\disorder{douleur} au \anatomy{thorax}''.
\end{itemize}


\subsubsection*{Discontinuous Entities}

Discontinuous entities are allowed if there is no easier solution, i.e.~when annotation as one entity is not possible or does not make sense.

Example with ``lung'':
\begin{enumerate}
    \item[1] ``it hurts in the \anatomy{left lung} and \anatomy{right lung}'' (two entities)
    
    \item[2] ``pain on the left lung, but no issues were found on the right'' $\rightarrow$ annotation using two separate, partially overlapping entities, one continuous and one discontinuous as shown below:
    \begin{itemize}
        \item ``\disorder{pain} on the \anatomy{left lung}, but no issues were found on the right'' (continuous)
        \item \drawRelations
            {\disorder{pain} \& on the left \& \anatomy{lung} \& , but no issues were found on the \& \anatomy{right}}
            {\discontRel{3}{5}}\\
            (discontinuous) 
    \end{itemize}

    \item[3] Similarly as above, the phrase ``nerves / muscle cramps'' is annotated as two separate, partially overlapping entities:
    \begin{itemize}
        \item \drawRelations
            {\disorder{nerves} \& / muscle \& \disorder{cramps}}
            {\discontRel{1}{3}}
            (discontinuous)
        \item ``nerves / \disorder{muscle cramps}'' (continuous)
    \end{itemize}


\end{enumerate}



\subsubsection*{Punctuation}

Punctuation is not to be included in the annotation (except for, e.g., hyphens or abbreviations etc.).

\subsubsection*{Spelling Mistakes / Slang}

Since spelling mistakes can occur often particularly when writing medication names, we treat them as if correctly written as long as we can easily understand what the user meant.

\subsubsection*{Abbreviations}

Further, also abbreviated expressions are to be annotated.
Abbreviations are often used for drug names, but also for disorders or doctor's names.

\examples{ex:abbr_general}
    {(\ldots) I felt like I would imagine someone that had \disorder{MS} would feel like (\ldots)}
    % \disorder{swaying}, \disorder{pulling myself up stairs,} \disorder{dragging legs}.}
    {Pour l'instant le \doctor{gastro} ne me parle pas de \other{rémission} !}
    {Ich glaube, ich bleibe auch erst einmal bei dem \doctor{Gyn} der letzten Jahre, (\ldots).}
    {*\drug{ドセ}の\disorder{浮腫み}も気になる} % I'm concerned about swelling by doce(taxel)

\subsection{Drug}\label{ent_drug}

With \dr, we annotate any mention of a medication name, brand, or agent.
We also include dietary supplements.

% {\color{blue} [Original]
% However, we \textit{do not} annotate other treatments, for instance, ``chemotherapy''.
% These occurrences are annotated as \texttt{other}.
% }

% {\color{red} [Proposal (shuntaro)]
As an exception, we include drug-based treatments too, such as ``chemotherapy'' and ``PUVA therapy'' (PUVA = \textbf{P}soralen and \textbf{U}ltra\textbf{v}iolet \textbf{A}).
However, we label completely non-drug involving therapies \texttt{other} instead.
% }

\examples{ex:drug}
    {\drug{Lipidor} \evalPos{did the job} on my \function{cholesterol} both LDL and HDL.}
    {l'\drug{entyvio} c 'est pas miracle mais bon \evalPos{ça aide}. bon courage à tous.}
    {Ich nehme \drug{Betablocker}.}
    {\timexPIT{夜}は\drug{睡眠導入剤}で\function{眠れている}んですけど、(\ldots)} % I can sleep at night with a sleep aid,
    
\subsubsection*{Abbreviations}

We also annotate abbreviations or colloquial names of drugs. 

\examples{ex:drug_abbr}
    {*In my desperation, I also tried an \drug{AD}.}
    % {des \disorder{douleurs} dans mes \anatomy{pouces} surtout celui de gauche, 
    {(\ldots) je passe une \test{radio} des \anatomy{2 mains} \timexPIT{mercredi matin}, avant ma \drug{perf}, (...)}
    % {Hatte in meiner Verzweiflung ja auch ein \drug{AD} ausprobiert.}
    {Nun habe ich also mit \drugStop{Candesartan} und \drugStop{Opi} (\ldots) \trigger{aufgehört}.}
    {なんか\drug{SSRI}増やしてほしい。} % I need more SSRI (lit.: (I) want (the doctor) to increase my SSRI)
    
\subsubsection*{Mentions referring to a drug}

Further, mentions referring to a drug but not clearly stating the drug's name are also annotated (see also example \ref{ex:disorder} (en)).

\examples{ex:drug_ref}
    {The \drug{pill} \evalPos{does its job}, (...)}
    {
    % (\ldots), j'ai pris \measure{deux} \drug{pilules} \timexDate{par jour}, (\ldots)
    \drawRelations{
    (\ldots), j'ai pris \& \measure{deux} \& \drug{pilules} \& \measure{par jour}\&, (\ldots)}
            {
            \discontRel{2}{4}
            }
    }
    {Ich nehme die \drug{Tabletten} \timexDur{seit 2 Tagen}.}
    {\drug{漢方薬}を購入して\trigger{飲み始めた}所、(\ldots)} % Once I started taking purchased Chinese medicine, ...


\subsection{Disorder}\label{ent_disorder}

A \dis annotation denotes any disease, sign or symptom related to the patient's health, including mental issues.
Sometimes a \dis may be expressed as a parameter in combination with a value: e.g., \disorder{high LDL} (parameter=LDL, value=high)\footnote{If the value does not directly modify the target noun, the corresponding appropriate entities should be labeled separately. E.g. ``\function{LDL} was \measure{high}''}. 
When the value is outside the normal range, this describes a \dis.
Sometimes, disorders are only referred to very broadly, e.g., it might happen that the patient simply says ``I do not feel well'';
These expressions are also treated as disorders.
%If there are hypernyms (e.g.~ ``experience'' for a list of different disorders)


\examples{ex:disorder}
    {I \trigger{tried} the advertised \drugStop{Arthritis medicines} with \disorder{severe side-effects} and only tried this one because the \doctor{doctor} had samples.}
    {J'ai une \disorder{maladie de crohn} \timexDur{depuis 36 ans}~ (\dots)}
    {\timexDur{Nach langen, qualvollen 11 Monaten} wurde eine \disorder{Gürtelrose ohne Ausschlag} diagnostiziert.}
    {\timexPIT{昼間}の\drug{レクサプロ}が\disorder{副作用}\evalNeg{ひどくて}未だに\disorder{気持ち悪い}} % Daytime Lexapro still makes me sick because the side effects are so bad.


\subsubsection*{Disorder vs. Function}

We annotate adverse biological processes as \dis, and neutral/positive processes as \texttt{function}, i.e.,~a negated \texttt{function} is a \dis (but not vice versa).

\examples{ex:dis_vs_func}
    {(\ldots) \disorder{could not urinate} (\ldots)}
    {nez entièrement bloqué au point d'avoir du \disorder{mal à respirer} mais \ldots}
    {Entweder \disorder{liegt man schlaflos wach} oder man wacht nach ein paar Stunden auf und  \disorder{kann nicht mehr schlafen}.}
    {\drug{抗うつ剤}だからやっぱり\disorder{記憶力の低下}が見られるそうですね。} % I hear that because it's an antidepressant, you still see memory loss.

\subsection{Anatomy}\label{ent_anatomy}

With this, we annotate all organs or anatomical parts.
We usually do not annotate smaller parts such as partial tissues and blood cells as \texttt{anatomy}, but as \texttt{function}. 
However, if a sentence describes a disorder found in a cell, the cell could be an \texttt{anatomy} entity.


\examples{ex:anatomy}
    {Had numerous odd \disorder{aches}, especially in the \anatomy{leg area}.}
    % {\ldots, des \disorder{boutons} \anatomy{de la tête au pied}, \ldots}
    {(\ldots) mais elle a fait une \disorder{réaction} au \anatomy{pancréas}, (\ldots).}
    {Ich besitze nur noch eine \anatomy{Niere}.}
    {**\anatomy{お腹}が\disorder{刺すように痛い}} % I feel a stabbing pain in my stomach (範囲は全体を対象とする:e.g. お腹から頭にかけて)


\noindent \texttt{Anatomy} entities are \textit{not annotated} when within a larger entity such as within a \dis or \texttt{test}.
Therefore, we prioritize the annotation of \dis and \texttt{test}.

\examples{ex:anatomy_embedded}
    {\disorder{Pain} under \anatomy{ribs} , \disorder{restless legs} .}
    {}
    {\test{Magen/Darmspiegelung}~, weil ich immer \disorder{Magenschmerzen} hatte.}
    {**\disorder{肺がん}と宣告されてしまった} % I was diagnosed with lung cancer.


\subsection{Test}\label{ent_test}

With \texttt{test}, we mark all medical tests, interviews, examinations or any other procedure that produces a result to be used in medical diagnoses.

\examples{ex:test}
    {\test{Blood test} was \measure{normal}.}
    % {des \disorder{douleurs} dans mes \anatomy{pouces} surtout celui de gauche, 
    {je passe une \test{radio} des \anatomy{2 mains} \timexPIT{mercredi matin}, avant ma \drug{perf}, (...)}
    {Ich habe ganz \timexPIT{zu Beginn}~ der \function{WJ} auch mal \test{Speicheltests} machen lassen, mit zum Teil abstrusen Werten.}
    {**\test{血液検査}の結果が怖いです} % I'm afraid of the blood test results.
    
    
\subsection{Function} \label{ent_function}

With \texttt{function}, we mark all body functions and processes.
This includes mental functions, too. 

%(e.g. ``I feel \function{anxiety} about how long this \disorder{pain} continues'') ; feeling anxiety against a pain seems a working mental function although it should be a \texttt{disorder} if it appeared as a mental issue). ← not applicable. negative mental function should be disorder always.

\examples{ex:function}
    {\test{EKG} \measure{perfect}, \test{tri's} were \measure{perfect}, \function{blood sugar} \measure{perfect}, etc }
     {}
    {\drugStop{Opripramol} \trigger{hatte} ich ja nur \timexDur{zwei Abende lang}~zum \function{Schlafen} je \measure{eine halbe} genommen, also nur eine \measure{winzige Dosis}.}
    {**\function{白血球}は問題なかった。} % no problem with leukocyte

However, as stated above, adverse physiological processes are annotated as \dis and therefore, only \emph{working} body functions are annotated as \texttt{function}.
{\color{red}
(!!) Japanese text needs ``negated functions'' because a style ``[function noun] is [negative adj/verbal-noun]'' often appears.
Without ``negated function'', we have to annotate the \textbf{whole sentences} as \dis.
\bigskip

I have no appetite: 
\begin{tabular}{c|c|c|c}
    まったく & 食欲 & が & ない \\
    at all & appetite & is & nothing \\
    (adv) & (noun) & (verb) & (adj) \\
\end{tabular}\bigskip

decreased white blood cell count: 
\begin{tabular}{c|c|c}
    白血球数 & が & 低下 \\
    \# of WBC & is & fall \\
    (noun) & (verb) & (verbal-noun) \\
\end{tabular}

}

\subsection{Opinion} \label{ent_eval}
This denotes the patient's \textit{personal} evaluation or opinion of a (\dr), health state (\dis) or biological process (\texttt{function}).
This entity could be a detailed description of a mental function or disorder (e.g. ``My \function{feeling} is \evalPos{stable}'').

% ``the \disorder{side effect} was \evalNeg{so terrible}''
% I'm so sad that I lost my appetite.

Usually, these assessments are rather colloquial and it is difficult to find an appropriate span.
Therefore, we follow again the principle of taking the shortest span possible.

The evaluation assessment always comes with an attribute: either \texttt{positive}, \texttt{negative}, or \texttt{neutral}.
A \texttt{positive} assessment is often associated with an improvement of a disease or with a good experience of the patient with medication.
We do not give \textit{negated} (the \textit{negation} attribute) to the corresponding \texttt{disorder}.

\examples{ex:eval}
    % {*I also \trigger{had} \drugStop{antibiotics} but that \evalNeut{didn't change anything}.\\ Then, after \trigger{starting} \drugStart{Trisequen's hormone pills}, just \timexDur{1 week later}, I was \evalPos{like in 7th heaven}.}
    % {\drug{Lipidor} \evalPos{did the job} on my \function{cholesterol} both LDL and HDL.}
    {*This \drug{drug} is a \evalNeg{nightmare} and \evalNeg{should be discontinued}. \\
    It \evalNeg{isn't working}.}
    {l'\drug{entyvio} c 'est pas miracle mais bon \evalPos{ça aide}. }
    {Ich nehme doch jetzt \timexDur{12 Wochen}~\trigger{kein} \drugStop{Lyrica} \trigger{mehr} und \trigger{dafür} \measure{50mg} \drugStart{Opi} und das \evalPos{lief auch richtig gut}.}
    {*\drug{アモキサン}\evalPos{効いてる}おかげか\function{気持ち}は\evalPos{少し楽になった}} % I'm feeling a little better now that the amoxan is working.
    
    
\subsubsection*{Sentiment}

\subsection{Measure} \label{ent_measure}

With \texttt{measure}, we mark clinically relevant measurements, such as drug dosages and test results.
The expression is typically a numerical value that accompanies a measurement unit.
{\color{red}
Note that, however, temporal measurements (e.g. ``5 times per month'') should be annotated as \texttt{time} entities (see the next section) unless they indicate the amount/dosage of a drug (``I took \measure{\textcolor{white}{three days' worth}} of \drug{medicines} at once \timexDate{today}.'', and ``Mom should take \drug{this pill} \measure{\textcolor{white}{three times a day}}'').
}

\examples{ex:measure}
    {Started \timexPIT{2 years ago}~ with \measure{10 mg} then \timexDur{6 mos later}\\ \doctor{doc} upped to \measure{20}.}
    % {\ldots, bon ça fait beaucoup, \timexDur{depuis 1 an et demi} ,perfusions d entyvio tous les mois. 
    {\drug{pentasa}, \drug{cortancyl}  (\measure{10mg}), \ldots}
    {Das \drug{Utrogest} sind \route{Weichkapseln} mit \measure{100 mg} \drug{naturid.~Progesteron} und sie sind verschreibungspflichtig.}
    {\test{生検}の結果\test{Ki67}が\measure{46%}だったため、通院にて治療中です} % The biopsy results showed a Ki67 of 46%, and I am in the hospital for treatment. 

\subsection{Time} \label{ent_timex}

As temporal markers we define all mentions of \textit{frequency}, \textit{duration}, \textit{date}, or \textit{relative points in time}. 
For a more narrow description, we use those characteristics (frequency, duration, \ldots) as attributes.
Also, we include prepositions (e.g. ``in'', ``from'', ``before'', and ``since'') in the entity since these carry relevant information specifying the semantics of the expression.
If there is no suitable attribute, we leave it blank.

\textcolor{red}{``\timexDate{on around February (in) three years ago}'' (3年前の2月ごろに)? or split this into two parts?}

Time expressions include e.g.~``night'', ``afternoon'', ``for one week'' (mostly \texttt{duration}), but also expressions like ``last Monday'', ``in two weeks'' (\texttt{relative points in time}), ``every morning'', ``after lunch'' (\texttt{frequency}) or ``11.07.2022`` (\texttt{date}).

\textcolor{red}{``about \timexDate{two years ago}''? or ``\timexDate{about two years ago}''?}

\examples{ex:time}
    {{\color{red} Instead of taking the \drugDec{pill} \measure{2X daily}, as prescribed I take it (\ldots) usually \measure{about 3-4 times per week}, but often skipping \timexFreq{a week or two at a time}.}}
    {Bonjour, Je suis sous \drug{entyvio} \timexDur{depuis juillet 2019}~et ce traitement\\ \evalPos{fonctionne fort bien} au niveau de la \disorder{colite}. }
    {Hallo liebe $<$user$>$, \timexPIT{vor 2 Monaten}~habe ich meine \doctor{Frauenärztin} gefragt, da ich zwischendurch einen \test{Hormontest} wollte.}
    {\timexDate{3年前の2月ごろに}\disorder{乳がん}が発覚} % Breast cancer was discovered around February 3 years ago.


\subsection{Route} \label{ent_route}

\texttt{Route} annotates the means of medication intake, e.g.~via pills, via injection.
If a mention like ``pill'' refers to a drug (and not to the means of intake), it should be annotated as \dr.
\examples{ex:route}
    {It has not eliminated the need for \route{oral} \drug{pain meds} in all situatations but \evalPos{has helped}.}
    {\ldots, \timexDur{depuis 1 an et demi}, \route{perfusion} d \drug{entyvio} tous les mois. \ldots}
    {Ich habe das leider nur mit \route{Tabletten} gemacht (mir wollte keiner meiner \doctor{Ärzte} eine \route{Infusion} verschreiben).}
    {*現在\route{内服}しているのは\drug{レトロゾール}です} 

\subsection{Medication Status} \label{ent_drug_status}
Here, we annotate phrases describing or triggering a change in medication intake.
To specify the change, we add the attributes \texttt{start} (the trigger specifies that a medication was just started), \texttt{stop} (a medication was stopped), \texttt{increase} (the dosage was increased), \texttt{decrease} (the dosage was decreased), and \texttt{unique\_dose} (the medication was only given/taken once) \emph{to the \dr they refer to}.

\examples{ex:change}
    {I now have \trigger{increased} my intake of \drugInc{Vitamin C} to \measure{16,000 mg / day}, and we'll see how this works out.}
    {
        \drawRelations{*I took \& \measure{{three days' worth}} \& of \drugInc{medicines} \& \measure{at once} \& \timexDate{today}.}{
        \discontRel{2}{4}
        }
    }
    {}
    {Ich hatte \timex{vor einer Woche} die \drugDec{Betablocker} \trigger{reduziert} aber nur von einer \measure{23mg} \trigger{auf die Hälfte}.}
    {} 


\subsection{User} \label{ent_user}

Since social media users use creative names that are not necessarily easy to find using pre-processing tools, we mark ``left-over'' user names as well, to de-identify them afterwards if necessary.
The examples below were randomly generated.

\examples{ex:user}
    {*Dear \user{Leopard Footballer}, thank you for your message.}
    {Bonjour \user{Mûre Vive} pour ma part ils n'ont pas été efficace (\ldots)}
    {Liebe \user{Broccoli Klarinette} danke für deinen Bericht.}
    {\user{ゆーりん}さん、ありがとうございます!} 

\subsection{URL} \label{ent_url}
In case there are any URLs or e-mail addresses missed when de-identifying the data, we mark them to be removed later.

\examples{ex:url}
    {Go to \link{www.xxxxx.xx.xx} for more info.}
    {Si vous souhaitez discuter plus avant: \link{www.xxx xxxx xxx xxx}.}
    {Wenn du möchtest, geh doch mal auf die Seite \link{www.xxx xxxx xxx xxx}.}
    {} 

\subsection{Personal information} \label{ent_personal}

With this marker, we annotate all other personal information that needs to be removed before data publication (jobs, city, doctor/hospital' names ..).
This kind of annotation is only necessary if the de-identification failed.
The examples below were randomly generated.

\examples{ex:info}
    {**My name is \info{Emanuel Streich} and I am \info{16 years old}.}
    {}
    {*Ich wohne in \info{Lilly Schlitzer}, das kennst du sicher.}
    {**\info{伊藤晴美}、\info{大学2年生}です。} 


\subsection{Doctor} \label{ent_doctor}
We moreover annotate all medical job descriptions (e.g.~physician, dentist, nurse, (psycho) therapist \ldots).

\examples{ex:doctor}
    {I took \drugInc{lipitor} \measure{10mg} \timexDur{for only about two months},then \doctor{cardiologist} \trigger{increased} it to \measure{20mg}.}
    {( peut etre de \disorder{l'arthrose} dixit mon \doctor{docteur traitant}).......}
    {(...), hatte ich \timexPIT{gestern}~einen Termin bei meinem \doctor{Internisten}.}
    {**病院の\doctor{先生}がこの薬を\route{飲め}って言うんです。でも信用できません} 

% \subsection{EXPERIENCER} 
% The person experiencing the \sout{side effects} disorder or function.

% \textcolor{green}{ABANDON?}

% \textcolor{blue}{Find some other way to make a difference between (1) self-reported events and (2) general knowledge / hearsay? $\rightarrow$ postpone this annotation to a future task.}

\subsection{Other} \label{ent_other}
With \other{other}, we annotate all other entities that refer to a kind of treatment or medical event (pregnancy, wound etc.), but for which we do not have a category.
For example, since we do not have an entity for treatments that are not drugs (e.g., \textit{cognitive behavioral therapy}), we would annotate this term as \other{cognitive behavioral therapy}.
Possible examples in addition are clinical tools (e.g. ``syringe'') and medical devices (e.g. ``wig'', ``dental implant'', and ``pacemaker'')
\examples{ex:other}
    {Just \other{physical therapy} and \drug{pain medication}.}
    {Suite à ma \other{deuxième grossesse} j'ai souhaité passé sur un traitement à faire à domicile, (\ldots)}
    {Ohne BH fallen die \other{Implantate} runter.}
    {*\drug{抗がん剤}の\disorder{副作用}で\anatomy{髪}が\disorder{抜け}て、\other{ウィッグ}を着用していました。} % Wore a wig because of hair loss due to side effects of anti-cancer drugs


\section{Attributes \textcolor{red}{→ move to each entity's section}}\label{sec:attributes}

Attributes are used to add more precise semantics to some of the entity annotations.
In Table \ref{tab:attributes}, we show all available attributes.

\begin{table}[h]
\centering
\begin{tabular}{@{}lll@{}}
\toprule
\textbf{section} & \textbf{entity} & \textbf{attributes} \\ \midrule
\ref{sec:neg} & \texttt{disorder} & negated \\
\ref{sec:sent}& \texttt{evaluation} & positive; negative; neutral\\
\ref{sec:timex_attributes}& \texttt{time} & frequency; duration; date; relative point in time \\
\ref{sec:durg_change_attributes}& \textbf{\dr} & started; stopped; increased; decreased; unique dose  \\ \bottomrule
\end{tabular}
\caption{The available attributes of the entities \dis, \texttt{evaluation}, \texttt{time}, and \dr.}
\label{tab:attributes}
\end{table}

\subsection{Negation}\label{sec:neg}

Currently, we only apply the negation attribute to the entity annotation \dis, as negated \texttt{function} entities are annotated as \dis. 
A negation of drugs is not necessary, since we cover a stopped medication treatment using the \dr attributes.

Note that we do not give ``negated'' value to the disorders that do not completely disappear (e.g. ``a \disorder{long-running headache} was \evalPos{almost eased}'') because they still exist at least a little.
Most such cases would accompany \texttt{evaluation} entities, which are regarded as richer descriptions of the disorders' status.

% \examples{ex:neg_attribute}
%     {The \disorderNeg{headaches} are gone.}
%     {}
%     {\timexDur{Seit zwei Tagen}~ist meine \disorderNeg{Übelkeit} wieder weg, doch da ...}
%     {} 
    
% If an entity mention is negated, we add a negation attribute to the entity (see Section~\ref{sec:attributes}). 
% The negation clue itself is not included in the span of the entity mention.
% Currently, only \dis entities can carry a negation attribute.
% We provide more details on negation in section \ref{sec:neg}.

\examples{negation}
    {There is no \disorderNeg{abnormality}.}
    {Par contre j'étais pas plus \disorderNeg{fatigué} que la normale.}
    {Folgende Beschwerden sind schon weg: \disorderNeg{Nackenschmerzen}, ...}
    {*\disorderNeg{イライラ}などはしなかった} % I didn't feel irritated or something



\subsection{Sentiment}\label{sec:sent}

When patients describe their current state of health or when they assess the consequences of a medication they took, they often use emotional words.
Therefore, we add, if applicable, the sentiment markers \texttt{positive}, \texttt{negative}, or \texttt{neutral} to the phrases annotated with \texttt{eval}.

\examples{ex:sentiment}
    {I really made a difference, \evalPos{improvement of quality of life}.}
    {Bonjour \user{xxxxxxxx} j'ai \evalPos{très bien toléré} \drug{entyvio} \evalPos{aucun effet indésirable}.}
    {Dann, nachdem ich mit \drugStart{Triseqenz Hormontabl.} \trigger{angefangen} habe, nur \timexDur{1 Woche später}~war ich \evalPos{wie im 7. Himmel}.}
    {} 

\subsection{Temporality}\label{sec:timex_attributes}

The following attributes of timex expressions are considered in our annotations:
\texttt{frequency}, \texttt{duration}, \texttt{date} and \texttt{relative point in time}. 
The attributes should help to specify the annotated timex expressions in more detail. See examples in section \ref{ent_timex}.
If none of the attributes fits, we do not need it.
% \examples{ex:att_timex}
%     {}
%     {}
%     {}
%     {} 

\subsection{Drug Changes}\label{sec:durg_change_attributes}

The attributes referring to the entity \dr provide additional details about the context in which the medication is taken. We consider the attributes \texttt{started}, \texttt{stopped}, \texttt{increased}, \texttt{decreased} and \texttt{unique dose}. Some examples are provided below:

\examples{ex:att_drug}
    {I now have \trigger{increased} my intake of \drugInc{Vitamin C} to \measure{16,000 mg / day}, and we'll see how this works out.}
    {}
    {Ich hatte \timexPIT{vor einer Woche}. die \drugDec{Betablocker} \trigger{reduziert} aber nur von einer \measure{23mg} \trigger{auf die Hälfte}.}
    {} 



\section{Relations}\label{sec:relations}

In the following, we will give details on how we annotate the relations between the entities.
For a better visualizations, we highlight only the relations and entities of the current section.
Also, we cut off parts of the original documents to keep every example on one line.
In Table \ref{sec:relations} we provide an overview of all relations and their arguments.
Exceptionally, \other{other} can take and accept all relations, for a practical reason.


\begin{table}[h]
\centering
\begin{tabular}{@{}llll@{}}
\toprule
\textbf{section} & \textbf{relation} & \textbf{argument 1} & \textbf{argument 2}\\ \midrule
\ref{rel_caused} & \colorbox{brass}{\texttt{caused}} & \{\dr, \dis\} & \{\dis\} \\
\ref{rel_treat} & \texttt{treatment\_for} & \{\dr\} & \{\dis, \texttt{function}\} \\
\ref{rel_dosage} & \texttt{dosage} & \{\dr\} & \{\texttt{measure}\} \\
\ref{rel_experienced} & \colorbox{airforceblue}{\texttt{experienced\_in}} & \{\dis\} & \{\texttt{anatomy}\} \\
\ref{rel_examined} & \texttt{examined\_with} & \{\dis\} & \{\texttt{test}\} \\
\ref{rel_refers} & \texttt{refers\_to} & \{\dis\} & \{\dis\} \\
 & \texttt{refers\_to} & \{\dr\} & \{\dr\} \\
\ref{rel_resulted} & \texttt{resulted\_in} & \{\texttt{test}\} & \{\texttt{measure}\} \\
\ref{rel_interacted} & \texttt{interacted\_with} & \{\dr\} & \{\dr\} \\
\ref{rel_change} & \texttt{change} & \{\texttt{drug\_status\_change\_trigger}\} & \{\dr\} \\
\ref{rel_time} & \texttt{time} & \{\texttt{time}\} & \{\dr, \dis\} \\
& \textcolor{red}{\texttt{misc}} & \{\texttt{ANY}\} & \{\texttt{ANY}\} \\
& \textcolor{red}{not tracked} & \textcolor{red}{\textbf{route}}, URL, personal info, doctor & \\
\bottomrule
\end{tabular}
\caption{Overview of available relations and the entities they associate.}
\label{tab:relations}
\end{table}


\textcolor{green}{
\begin{enumerate}
% \item link all mentions sharing a concept (those with the same surface form do not need to be linked (e.g.~ Lipitor \& Lipitor)
    \item link super/subordinate/generic concepts? 
    \item use ``chaining rule'' to link same concepts
\end{enumerate}
}


\subsection{Caused}\label{rel_caused}

\textcolor{green}{Patrick's proposal}

The \texttt{caused} relation was defined to be language independent and to be used also with little linguistic knowledge.
We only annotate a \texttt{caused} relation, when the entities \dr, \dis, or \texttt{function} are concerned. 
For this, we take into account the following:
\begin{itemize}
    \item The interpreted intention of the author if the annotator(s) judges it to be clearly identifiable.
    \item Common world knowledge (e.g., the action -- reaction principle) and common sense (e.g., a broken leg can not be caused by brushing one's teeth).
    \item A possible contextual cue from meta data (e.g., the meta data might contain the drug name, which is not necessarily mentioned in the text itself).
    \item Explicit formulation of a \cause-- \conseq relation, i.e., supported by linguistic markers like:
    \begin{itemize}
        \item explicit discourse inter-clause/sentence articulators: \textit{because}, \textit{so}, \textit{then}, \ldots (including conjunction coordination hinting at  causal entailment like ``and'', ``then'') %      <cause clause> then <consequence clause>
        \item conditional constructions: ``when \cause , \conseq'',  \\ ``\cause and \conseq'', \ldots
        \item restrictive formulations, e.g., ``only when \cause , \conseq''
    \end{itemize}
    \item Lexical semantics of nouns or verbs, e.g.,:
    \begin{itemize}
        \item \cause  provokes \conseq
        \item \conseq is the consequence of \cause
        \item \cause was followed by \conseq
        \item \cause entailed \conseq
        \item \conseq is correlated with \cause
        \item \cause is correlated with \conseq and \cause is preceding \conseq
        \item \cause is probably linked with \conseq and \cause is preceding \conseq
    \end{itemize}
    \item ($R_0$ is a \texttt{refers\_to} relation) and (\conseq is the target of a \texttt{caused} relation) and (\cause is the source of a \texttt{caused} relation pointing  to the source of $R_0$)
    \item \cause \conseq relations are reported by the author of the text (first person) or attributed to another person by the author of the text (second/third person, like parents or siblings)
    \item Being part of a juxtaposition of linguistic element/clauses/sentences (but only if supported by other context, either cotext and/or metadata elements):
    \begin{itemize}
        \item \cause \conseq
        \item \cause, \conseq
        \item \cause. \conseq.
    \end{itemize}
    \item Being part of a successive temporal relation
    \begin{itemize}
        \item \conseq after \cause
        \item before \conseq \cause
        \item every time \cause \conseq
    \end{itemize}
    \item Juxtaposition of \cause--\conseq clause and change of factuality / belief / veracity / opinion:
    \begin{itemize}
        \item \cause--\conseq belief clause  $<$veracity expression about previous clause$>$
        \item ``I though I would not have \conseq because of \cause. I was wrong!''
        \item ``I had never never thought that \cause would yield \conseq
        \item ``I would not have expected that \cause would yield \conseq
    \end{itemize}


 
\end{itemize}


However, we do \emph{not} annotate this relation if one of the following points apply:

\begin{itemize}
    \item The above mentioned entities are not concerned.
    \item The document at hand is hypothetical or speculative.
    \item The document is formulated as a question.
    \item The cause--consequence is not explicitly attributed to the author of the text, nor to a particular person or group of people identified via co-text or context (including the metadata of the document).
\end{itemize}


\examples{ex:caused}
    {\drawRelations{
        The \& \drug{pill} \& does \& its \& job \& , \& by \& I \& \disorder{constantly feel sick} \& to my stomach}
        {
        \causedRel{2}{9}
        }
    }
    {\drawRelations{
        Elle a pris \& \drugStop{infliximab} \& mais (\ldots) elle devient \& \disorder{tout rouge} \& et \& \disorder{nausée}}
        {
        \causedRel{2}{4} 
        \causedRel{2}{6}
        }
    }
    {\drawRelations{
        Hatte (\ldots) auch ein \& \drug{AD} \& ausprobiert. Bei mir hat es den \& \disorder{TSH hochgetrieben}!}
        {
        \causedRel{2}{4}
        }
    }
    {\drawRelations{
        \drug{レクサプロ} \& \trigger{飲み始めてから} \& \disorder{吐き気} \& はするし \& \disorder{食欲無い} \& し 
        }{
        \causedRel{1}{3}
        \changeRel{2}{1}
        \causedRel{1}{5}
        }
    } 

%Ensuite elle a pris infliximab mais au bout de 3 perfusion elle a développé des anticorps et grosse réaction allergique avec tachycardie importante et malaise, elle devient tout rouge et nausée, donc arrêt aussi.
% Hatte in meiner Verzweiflung ja auch ein AD ausprobiert. Bei mir hat es den TSH voll hochgetrieben!

\subsection{Treatment\_for}\label{rel_treat}

This relation connects a \dr and the targeted \dis. 

\examples{ex:treatment}
    {I have insomnia and fibromyalgia. The pain from the fibromyalgia is worsened when I can't sleep well.  My doctor prescribed this medication to me mainly for better sleep.}
    {atteinte de Crohn depuis 2017, actuellement sous perfusion entyvio depuis début 2019, le remsima ayant eu trop d'effets secondaires pour moi a  été abandonné.}
    {\drawRelations{habe \timex{vor 4-5 jahren} \& \drug{antibiotikum} und \drug{isotretinoin} wegen \disorder{starker akne} eingenommen}{\treatmentRel{2}{4}\treatmentRel{3}{4}}}
    {\drawRelations{
        \timexDate{先週から} \& \disorder{乳がん} \& の \& \drug{ホルモン剤} \& を \& \route{服用} \& \trigger{開始しました}
        }{
        \timeRel{1}{4}
        \treatmentRel{4}{2}
        \trigger{7}{4}
        }
    } 


\subsection{Dosage}\label{rel_dosage}

\texttt{Dosage} relates a measurement, e.g.~the number of pills (\texttt{measure}), with a medication name (\dr).

\examples{ex:dosage}
    {\drawRelations{
        I took \& \drugInc{lipitor} \& \measure{10mg} \& \timexDur{for only about two months}, \ldots
    % then \doctor{cardiologist} \trigger{increased} it to \& \measure{20mg}.}
        }{
        \dosageRel{2}{3}
        % \dosageRel{2}{5}
        }
    }
    {}
    {}
    {\drawRelations{
        \drug{エビリファイ} \& \timexDate{初日} \& ですが \& \measure{3mg} \& からなので \& \evalNeut{効き目はよくわかりません}
        }{
        \timeRel{2}{1}
        \dosageRel{4}{1}
        }
    } 
    

\subsection{Experienced\_in}\label{rel_experienced}

This relation associates a \dis with the location it is felt / experienced in, i.e.~a part of the body (\texttt{anatomy}).


    % {\drawRelations{
    %     Hatte (\ldots) auch ein \& \drug{AD} \& ausprobiert. Bei mir hat es den \& \disorder{TSH hochgetrieben}!}
    %     {
    %     \causedRel{2}{4}
    %     }
    % }

\examples{ex:experienced}
    {\drawRelations{
        \disorder{Pain} \& under \& \anatomy{ribs} \&, \disorder{restless legs} .}
        {
        \experiencedRel{1}{3}
        }
    }
    {}
    {}
    {\drawRelations{
        \anatomy{顔} \& と \& \anatomy{足} \& が \& \disorder{凄くむくむ}
        }{
        \experiencedRel{5}{1}
        \experiencedRel{5}{3}
        }
    } 

\subsection{Examined\_with}\label{rel_examined}

This relation associates a \dis with a \texttt{test}: the \texttt{test} is used to examine the \dis.

\examples{ex:examined}
    {}
    {}
    {}
    {\drawRelations{
        \timexDate{先月} \& の \& \test{検査} \& で \& \anatomy{肝臓} \& の \& \disorder{転移} \& を言われてしまいました
        }{
        \timeRel{1}{2}
        \examinedRel{7}{3}
        \experiencedRel{7}{5}
        }
    }    
    
    


\subsection{Refers\_to (formerly is\_identical)}\label{rel_refers}

The \texttt{refers\_to} relation represents a unidirectional link for the entities \dr, \dis, \texttt{anatomy}, and \texttt{function}.
It is used to associate similar concepts, but \emph{not} to connect the same surface forms (e.g.~ ``Lipitor'' and ``Lipitor''). 

Further, via the \texttt{refers\_to} relation, we can propagate the \texttt{caused} relation to all connected targets.

\begin{enumerate}
    \item \texttt{refers\_to} relations can be chained between corresponding entity mentions
    \item The first relation (\texttt{caused} / \texttt{treatment\_for}) is drawn to the concept the annotator thinks is most important to the understanding of the meaning of the sentences / document.
\end{enumerate}

\examples{ex:refers}
    {}
    {}
    {}
    {\drawRelations{
        \timexPIT{最近} \& \drug{抗うつ剤} \& の \& \drug{レクサプロ錠} \& \measure{10mg} \& も処方になった
        }{
        \timeRel{1}{2}
        \refersRel{2}{4}
        \dosageRel{5}{4}
        }
    }


    
\subsection{Resulted\_in}\label{rel_resulted}

Following the line of thought from above, the \texttt{resulted\_in} relation then relates the \texttt{test} with its result, usually a \texttt{measure}.
This relation should not be used for \texttt{caused} scenarios.

\examples{ex:resulted}
    {    
    \drawRelations{\test{Blood test} \& was \& \measure{normal}.}
    {
    \resultedRel{1}{3}
    }}
    {}
    {}
    {} 
    
\subsection{Interacted\_with}\label{rel_interacted}

In some cases, a \dr--\dr interaction results in a \dis. 
The relation \texttt{interacted\_with} is defined for exactly this purpose.

\examples{ex:interacted}
    {}
    {}
    {}
    {\drawRelations{
        * \& \drug{レクサプロ} \& と \& \drug{ビール} \& \route{飲んだら} \& \disorder{嘔吐} \& した
        }{
        \interactedRel{2}{4}
        \causedRel{2}{6}
        \causedRel{4}{6}
        }
    }

\subsection{Change}\label{rel_change}

For connecting the \texttt{drug status change triggers} with the \dr that is affected, we use the \texttt{change} relation.

\examples{ex:re_change}
    {}
    {\drawRelations{
        depuis que j'ai \& \colorbox{celadon}{arrêté$_{\texttt{trigger}}$} \& \drugStop{deroxat} \& \colorbox{celadon}{pour$_{\texttt{trigger}}$} \& \drugStart{effexor} \& j'ai perdu 3 kilos (\ldots)}
        {
        \changeRel{2}{3}
        \changeRel{2}{5}
        }
    }
    {}
    {\drawRelations{
        \drug{リフレックス} \& から \& \drug{レクサプロ} \& に \& \trigger{変えて} \& \disorderNeg{ふらつき} \& が \& 治まった    
        }{
        \changeRel{5}{1}
        \causedRel{1}{6}
        }
    } 

\subsection{Time}\label{rel_time}

The relation time connects a \texttt{time} expression with either a \texttt{drug} or a \texttt{disorder}.

\examples{ex:re_time}
    {}
    {\drawRelations{
    Bonjour, Je suis sous \&
    \drugStart{entyvio} \&
    \timexDate{depuis juillet 2019} \&
    et ce traitement (\ldots)
    }
    {\timeRel{3}{2}}
    }
    {}
    {\drawRelations{
        \drug{レクサプロ} \& \trigger{飲み始めて} \& \timexDur{2週間} \& たちました
        }{
        \changeRel{2}{1}
        \timeRel{3}{1}
        }
    } 
    
    
\section{Difficult Cases}

In the following sections, we show examples that we found difficult to annotate.
They are either ambiguous or it is in general not clear what is meant by the patient.

\subsection{English}

\begin{description}
    \item[Semantics] \drawRelations{
        \disorder{pain} \& on the \& \anatomy{left lung} \&, but \& \disorderNeg{no issues} \& were found on the \& \anatomy{right}}{
        \discontRel{3}{7}
        \experiencedRel{1}{3}
        \experiencedRel{5}{7}
        
        }

\end{description}

\subsection{French}

\subsection{German}

\subsection{Japanese}
    
\end{document}
