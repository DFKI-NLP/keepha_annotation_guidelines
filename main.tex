\documentclass[12pt]{article}
\usepackage[utf8]{inputenc}
\usepackage[T1]{fontenc}
\usepackage{amsthm}
\usepackage{url}
\usepackage[most]{tcolorbox}
\usepackage{colortbl}
\usepackage{philex}
\usepackage{booktabs}
\usepackage{tabu}
\usepackage{tikz-dependency}
\usepackage{ocgx}
\usepackage{tcolorbox}
\usepackage{tablefootnote}
\usepackage{float}
\usepackage{natbib}
\usepackage[a4paper, top=2.5cm, bottom=2.5cm, left=1.5cm, right=1.5cm]{geometry}


% ↓ this package enables Japanese-text input anywhere without using commands (added by Yada 2022-05-20)
\usepackage[whole]{bxcjkjatype}
\usepackage{ulem}
\usepackage{xspace}
\usepackage[hidelinks]{hyperref}


\usepackage{cleveref}

% FOR EMOJIS (Needs LuaLaTex)
% \usepackage{fontspec}
% \newfontfamily{\NotoEmoji}
%   {NotoColorEmoji.ttf}[Renderer=Harfbuzz]
% \newfontfamily{\SymbolaEmoji}{Symbola}

\tcbset{%any default parameters
  width=0.7\textwidth,
  halign=justify,
  center,
  breakable,
  colback=white    
}
\theoremstyle{definition}
\newtheorem{exmp}{Example}[section]

\xspaceaddexceptions{\}} % for \dis\}

% colors: https://latexcolor.com/
\definecolor{seabornOrange}{HTML}{DE8F05}
\definecolor{fuchsia}{HTML}{966FD6}
\definecolor{asparagus}{HTML}{006B3C}
\definecolor{ashgrey}{HTML}{B2BEB5}
\definecolor{babyblue}{HTML}{A1CAF1}
\definecolor{banana}{HTML}{FFA812}
\definecolor{bittersweet}{HTML}{FE6F5E}
\definecolor{brass}{HTML}{B5A642}
\definecolor{sienna}{HTML}{E75480}
\definecolor{burlywood}{HTML}{DEB887}
\definecolor{celadon}{HTML}{ACE1AF}
\definecolor{chestnut}{HTML}{CD5C5C}
\definecolor{cyan}{HTML}{008B8B}
\definecolor{dollarbill}{HTML}{85BB65}
\definecolor{heliotrope}{HTML}{DF73FF}
\definecolor{indianyellow}{HTML}{E3A857}
\definecolor{airforceblue}{HTML}{5D8AA8}

% The following are macros for ENTITY ANNOTATION
% generic entity
\newcommand{\entity}[1]{\colorbox{seabornOrange}{#1}}
% DRUG and different versions containing attributes
\newcommand{\drug}[1]{\colorbox{brass}{#1$_{\texttt{drug}}$}}
\newcommand{\drugStop}[1]{\colorbox{brass}{#1$_{\texttt{drug}}^{\textcolor{red}{stopped}}$}\ }
\newcommand{\drugStart}[1]{\colorbox{brass}{#1$_{\texttt{drug}}^{\textcolor{red}{started}}$}\ }
\newcommand{\drugInc}[1]{\colorbox{brass}{#1$_{\texttt{drug}}^{\textcolor{red}{increased}}$}\ }
\newcommand{\drugDec}[1]{\colorbox{brass}{#1$_{\texttt{drug}}^{\textcolor{red}{decreased}}$}\ }
\newcommand{\drugUD}[1]{\colorbox{brass}{#1$_{\texttt{drug}}^{\textcolor{red}{unique\_dose}}$}\ }
% ANATOMY
\newcommand{\anatomy}[1]{\colorbox{dollarbill}{#1$_{\texttt{anatomy}}$}\ }
% DISORDER & Negated DISORDER
\newcommand{\disorder}[1]{\colorbox{fuchsia}{\textcolor{white}{#1$_{\texttt{disorder}}$}}\ }
\newcommand{\disorderNeg}[1]{\colorbox{fuchsia}{\textcolor{white}{#1}$_{\textcolor{white}{\texttt{disorder}}}^{\textcolor{red}{negated}}$}\ }
% TEST & Negated Test
\newcommand{\test}[1]{\colorbox{asparagus}{\textcolor{white}{#1}$_{\textcolor{white}{\texttt{test}}}$}\ }
\newcommand{\testNeg}[1]{\colorbox{asparagus}{#1$_{\texttt{test}}^{\textcolor{red}{negated}}$}\ }
% FUNCTION
\newcommand{\function}[1]{\colorbox{banana}{#1$_{\texttt{function}}$}\ }
\newcommand{\functionNeg}[1]{\colorbox{banana}{#1$_{\texttt{function}}^{\textcolor{red}{negated}}$}\ }
% OPINION & positive/negative/neutral EVALUATION
\newcommand{\op}[1]{\colorbox{babyblue}{#1$_{\texttt{opinion}}$}\ }
\newcommand{\opPos}[1]{\colorbox{babyblue}{#1$_{\texttt{opinion}}^{\textcolor{red}{positive}}$}\ }
\newcommand{\opNeg}[1]{\colorbox{babyblue}{#1$_{\texttt{opinion}}^{\textcolor{red}{negative}}$}\ }
\newcommand{\opNeut}[1]{\colorbox{babyblue}{#1$_{\texttt{opinion}}^{\textcolor{red}{neutral}}$}\ }
% MEASURE
\newcommand{\measure}[1]{\colorbox{bittersweet}{#1$_{\texttt{measure}}$}\ }
% TIME expressions and corresponding attributes
\newcommand{\timex}[2]{\colorbox{ashgrey}{#1$_{\texttt{time}}$}}
\newcommand{\timexDur}[2]{\colorbox{ashgrey}{#1$_{\texttt{time}}^{\textcolor{red}{duration}}$}\ }
\newcommand{\timexFreq}[2]{\colorbox{ashgrey}{#1$_{\texttt{time}}^{\textcolor{red}{frequency}}$}\ }
\newcommand{\timexPIT}[2]{\colorbox{ashgrey}{#1$_{\texttt{time}}^{\textcolor{red}{rel.~point~in~time}}$}\ }
\newcommand{\timexDate}[2]{\colorbox{ashgrey}{#1$_{\texttt{time}}^{\textcolor{red}{date}}$}\ }
% ROUTE
\newcommand{\route}[1]{\colorbox{sienna}{#1$_{\texttt{route}}$}\ }
% TRIGGER
\newcommand{\trigger}[1]{\colorbox{celadon}{#1$_{\texttt{change\_trigger}}$}\ }
% OTHER
\newcommand{\other}[1]{\colorbox{burlywood}{#1$_{\texttt{other}}$}\ }
% USER
\newcommand{\user}[1]{\colorbox{chestnut}{#1$_{\texttt{user}}$}\ }
% URL
\newcommand{\link}[1]{\colorbox{cyan}{#1$_{\texttt{url}}$}\ }
% PERSONAL INFORMATION
\newcommand{\info}[1]{\colorbox{heliotrope}{#1$_{\texttt{perso\_info}}$}\ }
% DOCTOR
\newcommand{\doctor}[1]{\colorbox{indianyellow}{#1$_{\texttt{doctor}}$}\ }

% two short-cuts to write frequent terms in texttt mode
% \xspace inserts a space /if needed/
\newcommand{\dis}{\texttt{disorder}\xspace}
\newcommand{\dr}{\texttt{drug}\xspace}
\newcommand{\cause}{$<$\texttt{cause}$>$\xspace}
\newcommand{\conseq}{$<$\texttt{consequence}$>$\xspace}
% The following are macros for RELATION ANNOTATION:
% #1: argument 1
% #2: argument 2 of the relation 
\newcommand{\causedRel}[2]{\depedge[edge unit distance=0.8ex, label style={fill=yellow, font= \large}]{#1}{#2}{\texttt{caused}}}

\newcommand{\treatmentRel}[2]{\depedge[edge unit distance=0.5ex, edge height=4ex,  label style={fill=yellow, font= \large}]{#1}{#2}{\texttt{treatment\_for}}}

\newcommand{\dosageRel}[2]{\depedge[edge unit distance=0.5ex, edge height=4ex, label style={fill=yellow, font= \large}]{#1}{#2}{\texttt{has\_dosage}}}

\newcommand{\experiencedRel}[2]{\depedge[edge unit distance=0.5ex, edge height=4ex, label style={fill=yellow, font= \large}]{#1}{#2}{\texttt{experienced\_in}}}

\newcommand{\examinedRel}[2]{\depedge[edge unit distance=0.5ex, edge height=4ex, label style={fill=yellow, font= \large}]{#1}{#2}{\texttt{examined\_with}}}

\newcommand{\refersRel}[2]{\depedge[edge unit distance=0.5ex, edge height=4ex, label style={fill=yellow, font= \large}]{#1}{#2}{\texttt{refers\_to}}}

\newcommand{\resultedRel}[2]{\depedge[edge unit distance=0.8ex, edge height=4ex, label style={fill=yellow, font=\large}]{#1}{#2}{\texttt{has\_result}}}

\newcommand{\interactedRel}[2]{\depedge[edge unit distance=0.5ex, edge height=4ex, label style={fill=yellow, font= \large}]{#1}{#2}{\texttt{interacted\_with}}}

\newcommand{\changeRel}[2]{\depedge[edge unit distance=0.5ex, edge height=4ex, label style={fill=yellow, font=\large}]{#1}{#2}{\texttt{signals\_change\_of}}}

\newcommand{\timeRel}[2]{\depedge[edge unit distance=0.5ex, edge height=4ex, label style={fill=yellow, font=\large}]{#1}{#2}{\texttt{has\_time}}}

\newcommand{\evalRel}[2]{\depedge[edge unit distance=0.5ex, edge height=4ex, label style={fill=yellow, font=\large}]{#1}{#2}{\texttt{is\_opinion\_about}}}

\newcommand{\meansRel}[2]{\depedge[edge unit distance=0.5ex, edge height=4ex, label style={fill=yellow, font=\large}]{#1}{#2}{\texttt{has\_route}}}

\newcommand{\miscRel}[2]{\depedge[edge unit distance=0.5ex, edge height=4ex, label style={fill=yellow, font=\large}]{#1}{#2}{\texttt{misc}}}


\newcommand{\discontRel}[2]{\depedge[edge unit distance=0.5em, dashed]{#1}{#2}{\texttt{discont.}}}

% the RELATIONS environment; only the sentence chunks have to be defined in the respective texts
\newcommand{\drawRelations}[2]{
    \begin{dependency}[theme=default]
        \begin{deptext}
            #1 \\   % here goes the text, already split with "\&" into chunks
        \end{deptext}
        #2          % here go the relations (defined above), as many as you like, e.g. \causedRel{1}{5}, \causedRel{3}{8}
    \end{dependency}
}

% the EXAMPLE environment -- there is probably a much better solution.
\newcommand{\examples}[5]{%
    {\small
    \lb{#1}{
        \lba[en]{}{#2}
        \lbb[fr]{}{#3}
        \lbb[de]{}{#4}
        \lbz[ja]{}{#5}
    }}
}

\newcommand{\translated}{$^{\ast}$\xspace}
\newcommand{\madeup}{$^{\ast\ast}$\xspace}

% the translation environment; set to 0 to hide translations
% \newcommand{\translation}[1]{\begin{ocg}{translation}{translation}{1}\footnotesize{\\\begin{tcolorbox}[hbox]\textit{#1}\end{tcolorbox}}\end{ocg}}
\newcommand{\translation}[1]{\begin{ocg}{translation}{translation}{1}\footnotesize{\\\textit{[#1]}}\end{ocg}}

%%%%%%%%%%%%%%%%%%%%%%%%% MAIN DOCUMENT %%%%%%%%%%%%%%%%%%%%%%%%%%%%%%%%%%%%%%%
\title{KEEPHA Annotation Guidelines\\ Version 03/24}

% \author{Lisa Raithel$^\ast$, Shuntaro Yada$^\ast$, KEEPHA team}


\author{Lisa Raithel$^{\ast}$, Shuntaro Yada\thanks{Equal contribution; Corresponding author: raithel@tu-berlin.de},\\
      Hui-Syuan Yeh, Cyril Grouin, Thomas Lavergne,\\
      Aurélie Névéol, Patrick Paroubek, Philippe Thomas,\\
      Tomohiro Nishiyama, Sebastian Möller, Eiji Aramaki,\\
      Yuji Matsumoto, Roland Roller, Pierre Zweigenbaum}

% \address{$^1$BIFOLD, Ernst-Reuter Platz 7, 10587 Berlin, Germany;\\
%          $^2$Quality \& Usability Lab, TU Berlin, Ernst-Reuter Platz 7, 10587 Berlin, Germany;\\
%          $^3$German Research Center for Artificial Intelligence (DFKI), Alt-Moabit 91c, 10559 Berlin, Germany;\\
%          $^4$Université Paris-Saclay, CNRS, LISN, Rue du Belvédère, 91405 - Orsay, France;\\
%          $^5$Nara Institute of Science and Technology, 8916-5 Takayama-cho, Ikoma, Nara 630-0192, Japan;\\
%          $^6$RIKEN, Nihonbashi 1-chome Mitsui Building, 1-4-1 Nihonbashi, Chuo-ku, Tokyo 103-0027, Japan
%          }


% Lisa Raithel
% Shuntaro Yada
% Pierre Zweigenbaum
% Roland Roller
% Patrick Paroubek
% Aurélie Névéol
% Hui-Syuan Yeh

% Agata Savary
% Cyril Grouin

\date{\today}

\begin{document}

\maketitle

\setcounter{tocdepth}{2}
\tableofcontents
% \vspace{2cm}

\clearpage

\section{Overview}
In the following, we describe the annotation scheme we developed for our multi-lingual corpus KEEPHA.
The data for this corpus consists of different sources, for example, patient fora like Yahoo Q\&A or social media like Twitter (X).
% In Table \ref{tab:data}, we give an overview of the current annotated documents, their origin, and the language they are written in.
% Table \ref{tab:overview_detailed_annos} shows the number of annotations per language. 


% \begin{table}[h]
% \centering
% \begin{tabular}{@{}llll@{}}
% \toprule
% \textbf{name}   & {\textbf{language}}   & \textbf{\#documents}   & \textbf{source}\\ \midrule
% Lifeline-v1-de      & German (de)   & 118   &  www.lifeline.de  \\ 
% Lifeline-v1-fr      & French (fr)   & 100   &  www.lifeline.de (translated) \\
% lexapro\_ade        & Japanese (ja) & 560   & Twitter           \\
% chiebukuro          & Japanese (ja) & 59    & Yahoo! JAPAN Q\&A \\ \midrule
% Total               & de, fr, ja    & 837   &                   \\
% \bottomrule
% \end{tabular}
% \caption{Overview of the current KEEPHA dataset as of October 2023.}
% \label{tab:data}
% \end{table}

% \begin{table}[h]
% \centering
% \begin{tabular}{@{}lllll@{}}
% \toprule
% \textbf{language} & \textbf{\#documents} & \textbf{\#entities} & \textbf{\#relations} & \textbf{\#attributes}         \\ \midrule
% de & 118                  & 3,487               & 2,163                & 1,141             \\
% fr & 100                  & 1,939               & 1,129                & 537                \\ 
% ja & 619                   & 9,464               & 5,083                & 2,364                   \\ 
% \bottomrule
% \end{tabular}
% \caption{Number of documents (\#doc) in total per language with the number of entities, relations, and attributes.}
% \label{tab:overview_detailed_annos}
% \end{table}




Our annotation scheme includes entities, attributes, and relations. First, we define the annotated entities (Section \ref{sec:entities}), including possible attributes. Moreover, different examples of their usage are presented for each entity. 
Subsequently, we define the relations connecting these entities (Section \ref{sec:relations}).

Examples marked with \translated denote sentences that have been translated from a different language, and sentences marked with \madeup denote `artificial' examples that have been completely made up. All other examples are taken from existing corpora.
English examples are extracted from the \textsc{CADEC} corpus\footnote{\url{https://pubmed.ncbi.nlm.nih.gov/25817970/}} \citep{karimi_cadec_2015}, while German examples are taken from the German patient forum Lifeline\footnote{\url{https://fragen.lifeline.de/forum/}}.
French examples are translated from German using DeepL\footnote{\url{https://www.deepl.com/translator}} and manual verification.
% The French examples are taken from Lifeline-v1 documents, which were translated using DeepL\footnote{\url{https://www.deepl.com/translator}} and then manually verified and corrected if necessary.
Lastly, the Japanese examples are taken from social media (Twitter) and forum (Yahoo! JAPAN Q\&A) posts.
Note that sometimes, the translated sentences do not perfectly fit the original translations of the target languages. 
However, we keep the original annotation of the target language.

\begin{center}
\textcolor{blue}{You can toggle the translations of the non-English examples by clicking here\footnote{This feature is unfortunately not supported by all PDF viewers.}:}    
\end{center}

\begin{center}
\begin{tcolorbox}[hbox]
\switchocg{translation}{
\textcolor{blue}{\texttt{show/hide translations}}
}
\end{tcolorbox}
\end{center}


\begin{table}[htbp]
\centering
% \renewcommand{\arraystretch}{1.3}
\small
\begin{tabular}{@{}lll@{}}
\toprule
\textbf{section} & \textbf{entity}   & {\textbf{attributes}} \\ \midrule

\ref{ent_drug} &\drug{\dr}                          & \texttt{increase}, \texttt{decrease}, \texttt{stopped}, \texttt{started}, \texttt{unique\_dose} \\
\ref{ent_drug_status}&\trigger{\texttt{change trigger}}  & \\
\ref{ent_disorder}&\disorder{\dis}                     & \texttt{negated}\\
\ref{ent_function}&\function{\texttt{function}}        & \texttt{negated}\\                
\ref{ent_anatomy}&\anatomy{\texttt{anatomy}}          & \\              
\ref{ent_test}&\test{\texttt{test}}                & \\          
\ref{ent_eval}&\op{\texttt{opinion}}    & \texttt{positive}, \texttt{negative}, \texttt{neutral} \\
\ref{ent_measure}&\measure{\texttt{measure}}          & \\
\ref{ent_timex}&\timex{\texttt{time}}~             & \texttt{frequency}, \texttt{duration}, \texttt{date}, \texttt{point} \\
\ref{ent_route}&\route{\texttt{route}}              & \\
\ref{ent_doctor}&\doctor{\texttt{doctor}}            & \\
\ref{ent_user}&\user{\texttt{user}}                & \\
\ref{ent_url}&\link{\texttt{url}}                 & \\
\ref{ent_personal}&\info{\texttt{personal information}}   & \\
\ref{ent_other}&\other{\texttt{other}}              & \\

\bottomrule
\end{tabular}
\caption{Overview of the entities and their attributes.}
\label{tab:all_entities}
\end{table}



\section{Entities}\label{sec:entities}

In \Cref{tab:all_entities}, we show all available entities and their respective attributes.
Attributes are used to add more precise semantics to some of the entity annotations.

\subsection{General Guidelines for Entity Annotation}

We now define general rules for the annotation of entities.

\subsubsection*{Scope}

Regarding scope, we annotate entities in the form of noun or verb phrases together with their modifying parts, e.g., adjectives, adverbs, etc., though complex post modifiers (e.g., relative clauses: ``a pain that is \ldots'', and prepositional-phrase modifiers: ``a neck pain from yesterday'') are excluded.
We always prefer the smallest core noun phrase. 
If this does not work, a long span of the whole verb phrase (or even the whole clause/sentence) is allowed.

\examples{ex:scope}
    {I have \disorder{absolutely no appetite} and \disorder{constantly feel sick} to my \anatomy{stomach}.}
    {(\ldots) ça fait \disorder{drolement mal} et c'est \disorder{très gênant}. \translation{It \disorder{hurts awfully}, and it is a \disorder{great discomfort} }} 
    {(\ldots) ich hatte in der Situation auch \disorder{absolut keinen Hunger}. \translation{I was also \disorder{absolutely not hungry} in this situation.}}
    {\madeup \disorder{激しい嘔吐}に見舞われました \translation{I suffered \disorder{severe vomiting}.}} % I suffered severe vomiting.


\subsubsection*{Metaphors and descriptive language}

Further, we annotate descriptive expressions based on the \textit{patient's perspective} even if they do not occur in a medical dictionary.
We always try to follow the patient's perspective: If the patient thinks something (e.g., a symptom, an irregularity) is a disorder, then we annotate it as \dis.

\examples{ex:descriptions}
    {(\ldots) I felt like I would imagine someone that had \disorder{MS} would feel like, \disorder{swaying}, \disorder{pulling myself up stairs}, (\ldots)}
    {(\ldots) J'ai \disorder{failli tomber dans les pommes}, (\ldots) \translation{I \disorder{almost passed out}}}
    {Die \anatomy{Beine} \disorder{fühlen sich nicht stark} man hat \disorder{Gefühle des Ungleichgewichts}. \translation{The \anatomy{legs} \disorder{do not feel strong}, there are \disorder{feelings of imbalance}}}
    % {} % Zolpidem makes my brain shut down
    {ずっと\anatomy{お腹}が\disorder{ゴロゴロいってる} \translation{My \anatomy{stomach} \disorder{keeps saying grrrr}}} % 

%Seit zwei Tagen ist meine \entity{Übelkeit}$^{\attribute{negated}}$ wieder weg, doch da dieses Kopfkino mich vollkommen fertig macht und mir viel an Lebensqualität und -freude nimmt, .. || Ich habe nun das Problem, dass meine Haut sehr juckt. || Ich habe auch immer wieder mal seltsames Herzstolpern. Manchmal überschlägt sich das Herz, dann setzt es mal wieder kurz aus, manchmal hab ich für ein paar Sekunden ein Rasen, ||



Often, patients use metaphors to describe their suffering or, on the other side, a good experience with medication therapy.
Since metaphors are descriptive language as well, we annotate them as we would annotate explicit symptom descriptions.
Later, we can re-investigate those ``special disorders''.

\examples{ex:metaphors}
    {I felt always like I had a \disorder{veil over my head} and \disorder{lead in my legs}.}
    {(\ldots) à chaque effort, mon \anatomy{cœur} \disorder{bat la chamade} et jai ne n'ai \\ \disorder{pas beaucoup de souffle}. \translation{With every effort, my \anatomy{heart} is \disorder{racing} and I am \disorder{out of breath}}}
    {(\ldots) nur \timexPIT{1 Woche später}~war ich \opPos{wie im 7. Himmel}. \translation{Just \timexPIT{one week later}, I was \opPos{on cloud nine}.}}
    {\translated \drug{マイスリー}を飲むと\anatomy{脳}が\disorder{強制終了}する \translation{\drug{Zolpidem} makes my \anatomy{brain} \disorder{shut down}}}



\subsubsection*{Determiners and Possessives}

We do not include determiners or possessive pronouns in an entity.

\examples{determiners}
    {\translated I only really felt the \disorder{burning sensation} in the \anatomy{arms} and \anatomy{chest} when I ...}
    {(\ldots) J'ai développé une \disorder{très forte sinusite chronique},(\ldots) \translation{I have developed \disorder{severe chronic sinusitis}}}
    {\timexDur{Seit zwei Tagen}~ist meine \disorderNeg{Übelkeit} wieder weg, doch da (\ldots) \translation{\timexDur{Since two days}~ my \disorderNeg{nausea} is gone again, but since (\ldots)}}
    {\madeup この\drug{薬}は\opNeg{効かなかった} \translation{This \drug{drug} \opNeg{didn't work}.}}


\subsubsection*{Enumerations}

Consecutive entities or enumerations that are, for example, separated by commas or conjunctions are to be split apart and annotated separately.

\examples{enumerations}
    {I felt \disorder{pain} in my \anatomy{arms}, \anatomy{legs}, \anatomy{ears} and \anatomy{toes}.}
    {\measure{4} \other{opérations},  \drug{himurel}, \drug{pentasa}, \drug{methotrexate}, (\ldots) \translation{\measure{4} \other{surgeries},  \drug{himurel}, \drug{pentasa}, \drug{methotrexate}, (\ldots)}}
    {\timexPIT{Vor einem Jahr}~ bekam ich \timexFreq{kurz hintereinander} zwei \disorder{Herzinfarkte}, einen \disorder{Schlaganfall} und eine \disorder{Embolie}. \translation{\timexPIT{A year ago} I had two \disorder{heart attacks}, a \disorder{stroke} and an \disorder{embolism} in \timexFreq{quick succession}.}}
    {\anatomy{まつ毛}や\anatomy{眉毛}も\disorder{抜けてしまった}ので \translation{Because my \anatomy{eyelashes} and \anatomy{eyebrows} have \disorder{fallen out} too.}}

\subsubsection*{Nested and Overlapping Entities}

Nested entities are not permitted; we always annotate the largest phrase.
For example, in the case of ``headache,'' we do not split the word into ``head'' and ``ache''. 
The general rule is to apply a syntax-based decision, i.e., basic noun phrases (without prepositions, particles, etc.) are annotated as one entity.
This is also language-dependant:

\paragraph{German:} 
Do not split compound words, e.g., ``\disorder{Kopfschmerzen}'' \textit{[headache]} is annotated as one entity.
    
\paragraph{Japanese:} 
Do not split compound words based on standard tokenization, but split noun phrases (including compound nouns) at ``prepositions'' (particles て, に, を, は, の).
    
\paragraph{French:} Basic noun phrases (``\disorder{douleur thoracique}'') \textit{[chest pain]} are annotated as one \dis, otherwise we separate the annotations, e.g., ``\disorder{douleur} au \anatomy{thorax}'' \textit{[chest pain]}.

\medskip

% This approach also automatically excludes overlapping entities. 
For specific entity types, we provide prioritization rules to help in the decision of which mention to annotate.
The prioritization is mentioned for the entity types in their respective sections in these guidelines. 

\subsubsection*{Discontinuous Entities}

Discontinuous entities are allowed if there is no easier solution, i.e., when annotation as one entity is not possible or does not make sense. 
Please see the following examples for a comparison of different cases:

\begin{enumerate}
    \item ``it hurts in the left lung and right lung''\\
     $\rightarrow$ Here, two entities are annotated separately:
     \begin{itemize}
         \item ``it hurts in the \anatomy{left lung} and \anatomy{right lung}''
     \end{itemize}
    
    \item ``pain on the left lung, but no issues were found on the right''\\
    $\rightarrow$ The annotation uses two separate, partially overlapping entities, one continuous and one discontinuous, as shown below:
    \begin{itemize}
        \item ``pain on the \anatomy{left lung}, but no issues were found on the right'' (continuous entity)
        \item \drawRelations
            {pain \& on the  \& left \anatomy{lung} \& , but no issues were found on the \& \anatomy{right}}
            {\discontRel{3}{5}}\\
            (discontinuous entity) 
    \end{itemize}

    \item ``nerves / muscle cramps''\\
    $\rightarrow$ Similarly as above, the phrase ``nerves / muscle cramps'' is annotated as two separate, \textit{partially overlapping} entities:
    \begin{itemize}
        \item \drawRelations
            {\disorder{nerves} \& / muscle \& \disorder{cramps}}
            {\discontRel{1}{3}}
            (discontinuous entity)
        \item ``nerves / \disorder{muscle cramps}'' (continuous entity)
    \end{itemize}
\end{enumerate}

In case a discontinuous entity is necessary, we limit the number of entity fragments to two to reduce the burden for the annotators.

\examples{discontinuous}
    {\translated \drawRelations{(\ldots) although my \& \disorder{progesterone values} \& have been in the \& \disorder{lowest levels} \& for years.}{\discontRel{2}{4}}}
    {\drawRelations{(\ldots) et tous mes \& \disorder{symptômes} \& se sont à nouveau \& \disorder{intensifiés}, (\ldots)}{\discontRel{2}{4}} 
    \translation{(\ldots) and all my \disorder{symptoms} have \disorder{intensified} again, \ldots}}
    {\drawRelations{(\ldots) obgleich meine \& \disorder{Progesteronwerte} \& seit Jahren \& \disorder{im Keller} \& waren.}{\discontRel{2}{4}} 
    \translation{(\ldots) although my \disorder{progesterone values} have been in the \disorder{lowest levels} for years.}}
    {...}


\subsubsection*{Punctuation}

Punctuation markers are not included in the annotation (except for, e.g., hyphens, abbreviations, etc.).

\subsubsection*{Spelling Mistakes / Colloquial Language}

Since spelling mistakes can occur often, particularly when writing medication names, we treat them as if they were correctly written as long as we can easily understand what the user meant.

\subsubsection*{Abbreviations}

Further, abbreviated expressions are to be annotated as well.
Abbreviations are often used for drug names, but also for disorders or doctors' profession names.

\examples{ex:abbr_general}
    {(\ldots) I felt like I would imagine someone that had \disorder{MS} would feel like (\ldots)}
    % \disorder{swaying}, \disorder{pulling myself up stairs,} \disorder{dragging legs}.}
    {Pour l'instant le \doctor{gastro} ne me parle pas de \other{rémission} ! \translation{so far, my \doctor{GI} hasn't said anything about \other{remission} !}}
    {Ich glaube, ich bleibe auch erst einmal bei dem \doctor{Gyn} der letzten Jahre, (\ldots). \translation{I think I'll stay with my \doctor{gyn} of the last years for the time being.}}
    {*\drug{ドセ}の\disorder{浮腫み}も気になる \translation{I'm concerned about \disorder{swelling} by \drug{doce(taxel)}}} % I'm concerned about swelling by doce(taxel)

\subsection{Drug}\label{ent_drug}

With \drug{\dr}, we annotate any mention of a medication name, brand, or agent.
We also include dietary supplements.
% {\color{blue} [Original]
% However, we \textit{do not} annotate other treatments, for instance, ``chemotherapy''.
% These occurrences are annotated as \texttt{other}.
% }
% {\color{red} [Proposal (shuntaro)]
As an exception, we include drug-based treatments, too, such as ``chemotherapy'' and ``PUVA therapy'' (PUVA = \textbf{P}soralen and \textbf{U}ltra\textbf{v}iolet \textbf{A}).
Therapies that do not involve any drugs whatsoever are labeled as \texttt{other}.

\examples{ex:drug}
    {\drug{Lipidor} \opPos{did the job} on my \function{cholesterol} both LDL and HDL.}
    {l'\drug{entyvio} c 'est pas un miracle mais bon \opPos{ça aide}. bon courage à tous. \\ \translation{\drug{entyvio} is no miracle, but well, \opPos{it does help}. good luck to all.}}
    {Ich nehme \drug{Betablocker}. \translation{I take \drug{beta blockers}.}}
    {\timexFreq{夜}は\drug{デパス}で\function{眠れている}んですけど、(\ldots) \translation{I can \function{sleep} \timexFreq{at night}~ with \drug{Etizolam}}}


\paragraph{Prioritization of \dr} In case \texttt{route} or \texttt{anatomy} and \dr are one word (e.g., a compound) or connected by a hyphen, which might be the case for German, we prioritize \dr over \texttt{route} and label the entire spans as \dr.

\examples{prioritize_drug}
    {\translated Have the greatest respect for \drug{pills} in general, (\ldots)}
    {(\ldots) que tu prenais un peu de \drug{crème} le matin.}
    {``\drug{Urogestgel}'' or ``\drug{Estradiol-Creme}'' \translation{\drug{Urogest} gel / \drug{Estradiol} cream}}
    {吐き気を抑えるのに\drug{飲み薬}と\drug{胃薬}を使いました \translation{I used an \drug{internal medicine} and a \drug{stomach medicine} to control the nausea.}}

\subsubsection{Triggers for Medication Status Changes} \label{ent_drug_status}
Here, we annotate phrases describing or triggering a change in medication intake with the entity tag \trigger{\texttt{change\_trigger}}.
Typical words might be, for example, \textit{increase}, \textit{reduce}, \textit{begin}, \textit{one dose of}, etc.

To specify what kind of change occurred, we add the attributes \texttt{start} (the trigger specifies that medication was just started), \texttt{stop} (a medication was stopped), \texttt{increase} (the dosage was increased), \texttt{decrease} (the dosage was decreased), and \texttt{unique\_dose} (the medication was only given/taken once) \textit{to the} \dr \textit{they refer to} (see below in \Cref{sec:drug_change_attributes}).

\examples{ex:change}
    {I now have \trigger{increased} my intake of \drugInc{Vitamin C} to \measure{16,000 mg / day}, and we'll see how this works out.}
    {J'ai \trigger{oublié} de prendre mon \drugDec{citalopram} \timexDur{pendant 3 jours}\\\translation{I \trigger{forgot} to take my \drugDec{citalopram} \timexDur{for 3 days}}}
    {Ich hatte \timexPIT{vor einer Woche}~ die \drugDec{Betablocker} \trigger{reduziert} aber nur von einer \measure{23mg} \measure{auf die Hälfte}. \translation{I had \trigger{reduced} \drugDec{beta blockers} \timexPIT{a week ago}~ but only from \measure{23mg} to \measure{half of it}}}
    {\timexPIT{10日後くらいに}\drugStart{ドセタキセル}が\trigger{始まります} \translation{\drugStart{Docetaxel} \trigger{starts} \timexPIT{in about 10 days}.}}
    


\subsubsection{Attribute: Drug Changes}\label{sec:drug_change_attributes}

The attributes referring to the entity \dr provide additional details about the context in which the medication is taken. We consider the attributes \texttt{started}, \texttt{stopped}, \texttt{increased}, \texttt{decreased} and \texttt{unique dose}. Some examples are provided below:

\examples{ex:att_drug}
    {I now have \trigger{increased} my intake of \drugInc{Vitamin C} to \measure{16,000 mg / day}, and we'll see how this works out.}
    {Je suis sous \drug{sertraline} (\measure{25} \trigger{puis} \measure{50} en dosage) \\ \textit{[I've been taking \drug{sertraline} (dosage \measure{25} \trigger{then} \measure{50} )]} }
    {Ich hatte \timexPIT{vor einer Woche}. die \drugDec{Betablocker} \trigger{reduziert} aber nur von einer \measure{23mg} \trigger{auf die Hälfte}.}
    {\translated \disorder{副作用}が強くて、\drugStop{レクサプロ}\trigger{辞めました}! \translation{I \trigger{stopped} \drugStop{Lexapro} because a \disorder{side effect} was strong}} 

If the patient describes having forgotten the medicine, we add the attribute \texttt{decrease} except the context suggests a complete stop (then we add \texttt{stop}).
Conversely, if she describes taking, for example, a double dose, this will be marked as an \texttt{increase}.

\subsubsection*{Abbreviations of drug names}

We also annotate abbreviations or colloquial names of drugs. 

\examples{ex:drug_abbr}
    {*In my desperation, I also tried an \drug{AD}.}
    % {des \disorder{douleurs} dans mes \anatomy{pouces} surtout celui de gauche, 
    {(\ldots) je passe une \test{radio} des \anatomy{2 mains} \timexPIT{mercredi matin}, avant ma \drug{perf}, (...)
    \translation{I'm having a \test{CT} for \anatomy{both hands} on \timexPIT{Wednesday morning}, before my \drug{IV}}}
    % {Hatte in meiner Verzweiflung ja auch ein \drug{AD} ausprobiert.}
    {Nun habe ich also mit \drugStop{Candesartan} und \drugStop{Opi} (\ldots) \trigger{aufgehört}. \translation{So now I have \trigger{stopped} using \drugStop{Candesartan} and \drugStop{Opi}.}}
    {なんか\drug{SSRI}増やしてほしい。\translation{I need more \drug{SSRI}}} % I need more SSRI (lit.: (I) want (the doctor) to increase my SSRI)
    
\subsubsection*{Mentions referring to a drug}

Further, mentions referring to a drug but not clearly stating the drug's name are also annotated (see also Example \ref{ex:disorder} (en)).

\examples{ex:drug_ref}
    {The \drug{pill} \opPos{does its job}, (...)}
    {
    % (\ldots), j'ai pris \measure{deux} \drug{pilules} \timexDate{par jour}, (\ldots)
    \drawRelations{
    (\ldots), j'ai pris \& \measure{deux} \& \drug{pilules} \& \measure{par jour}\&, (\ldots)}
            {
            \discontRel{2}{4}
            } 
            \translation{\drawRelations{
                I took \& \measure{two} \& \drug{pills} \& \measure{a day}}
            {
            \discontRel{2}{4}
            }}
    }
    {Ich nehme die \drug{Tabletten} \timexDur{seit 2 Tagen}. \translation{I have been taking the \drug{pills} \timexDur{for two days}.}}
    {\drug{漢方薬}を購入して\trigger{飲み始めた}所、(\ldots) \translation{Once I \trigger{started} taking purchased \drug{Chinese medicine}, (\ldots)}}


\subsection{Disorder}\label{ent_disorder}

A \disorder{\dis} annotation denotes any disease, sign, or symptom related to the patient's health, including mental issues.
Sometimes a \dis may be expressed as a parameter in combination with a value: e.g., \disorder{high LDL} (parameter=LDL, value=high)\footnote{If the value does not directly modify the target noun, the corresponding appropriate entities should be labeled separately. Example: ``\function{LDL} was \measure{high}''}. 
When the value is outside the normal range, this describes a \dis.
Sometimes, disorders are only referred to very broadly, e.g., it might happen that the patient simply says ``I do not feel well'';
These expressions are also treated as disorders.
%If there are hypernyms (e.g.~ ``experience'' for a list of different disorders)


\examples{ex:disorder}
    {I \trigger{tried} the advertised \drugStop{Arthritis medicines} with \disorder{severe side-effects} and only tried this one because the \doctor{doctor} had samples.}
    {J'ai une \disorder{maladie de crohn} \timexDur{depuis 36 ans}~ (\dots) 
    \translation{I've had \disorder{crohn's disease} \timexDur{for 36 years}~ (\dots)}}
    {\timexDur{Nach langen, qualvollen 11 Monaten}~ wurde eine \disorder{Gürtelrose ohne Ausschlag} diagnostiziert. \translation{\timexDur{After a long, agonizing 11 months}, I was diagnosed with \disorder{shingles without a rash}.}}
    {\timexPIT{昼間}の\drug{レクサプロ}が\disorder{副作用}\opNeg{ひどくて}未だに\disorder{気持ち悪い} \translation{\timexPIT{Daytime} \drug{Lexapro} still \disorder{makes me sick} because the \disorder{side effects} are so \opNeg{bad}}}


\subsubsection*{Disorder vs. Function}\label{sec:dis_vs_func}

We annotate adverse biological processes as \dis and neutral/positive processes as \texttt{function}, i.e., in general, a negated \texttt{function} is a \dis (but not vice versa).
To distinguish \dis versus \texttt{function}, we apply the following rules:

\begin{enumerate}
    \item If the entity in question is a noun phrase and describes a malfunction, we label it as a \dis rather than a negated function:
        ``I have \disorder{no appetite}.''
    \item If, however, a malfunction is expressed beyond a continuous noun phrase, we label it negated function:
        ``My \functionNeg{appetite} has disappeared.''
\end{enumerate}


\subsubsection{Negation of Disorders}\label{sec:neg_dis}

Currently, we only apply the negation attribute to the entity annotation \dis and \texttt{function}. %as negated \texttt{function} entities are annotated as \dis. 
% A negation of drugs is not necessary since we cover a stopped medication treatment using the \dr attributes.
For \dis, note that we do not assign the ``negated'' attribute to the disorders that do not completely disappear (e.g., ``a \disorder{long-running headache} was \opPos{almost eased}'') because they still exist at least a little.
Most such cases would accompany \texttt{opinion} entities, which are regarded as richer descriptions of the disorders' status.

\examples{negation}
    {There is no \disorderNeg{abnormality}.}
    {Par contre j'étais pas plus \disorderNeg{fatigué} que la normal.
    \translation{however, I wasn't more \disorderNeg{tired} than usual.}}
    {Folgende Beschwerden sind schon weg: \disorderNeg{Nackenschmerzen}, (\ldots )
    \translation{The following ailments are already gone: \disorderNeg{neck pain}, }}
    {*\disorderNeg{イライラ}などはしなかった 
    \translation{I didn't \disorderNeg{feel irritated} or something}} % I didn't feel irritated or something

\subsubsection{Partial Negation (of Disorders)}\label{sec:partial_negation}

We \textit{do not} annotate partial negations since these require a more detailed annotation, which is, for now, out of scope.

\examples{ex:partial_negation}
    {\translated My \anatomy{hair} has \emph{not completely fallen out}.}
    {\translated Les \disorder{attaques de panique} ont \emph{presque disparu}.
    \translation{The \disorder{panic attacks} are \emph{almost gone}.}}
    {Die \anatomy{Brust} tut schon \emph{viel weniger weh}. 
    \translation{The \anatomy{chest} \emph{hurts much less} already.}}
    {\anatomy{髪}が\emph{完全に抜けたわけではありません} 
    \translation{My \anatomy{hair} has \emph{not completely fallen out}.}}

% \examples{ex:dis_vs_func}
%     {(\ldots) \disorder{could not urinate} (\ldots)}
%     {nez entièrement bloqué au point d'avoir du \disorder{mal à respirer} mais \ldots \translation{my \anatomy{nose} is totally congested, I \disorder{ can barely breathe} but \ldots}}
%     {Entweder \disorder{liegt man schlaflos wach} oder man wacht nach ein paar Stunden auf und \disorder{kann nicht mehr schlafen}. \translation{Either you \disorder{lie awake sleepless} or you wake up after a few hours and \disorder{can't sleep anymore}.}}
%     {\drug{抗うつ剤}だからやっぱり\disorder{記憶力の低下}が見られるそうですね。\translation{I hear that because it's an \drug{antidepressant}, you still see \disorder{memory loss}.}} % I hear that because it's an antidepressant, you still see memory loss.

\subsection{Function} \label{ent_function}

With \function{\texttt{function}}, we mark all body functions and processes.
Body functions are often represented in biomarkers (e.g., ``HDL'' and ``white blood cells (WBC)'').
This includes mental functions, too. 

%(e.g. ``I feel \function{anxiety} about how long this \disorder{pain} continues'') ; feeling anxiety against a pain seems a working mental function although it should be a \texttt{disorder} if it appeared as a mental issue). ← not applicable. negative mental function should be disorder always.

\examples{ex:function}
    {\drug{Lipidor} \opPos{did the job} on my \function{cholesterol} both LDL and HDL.}
    {La seule chose qu'on m'a donné pour essayer de \function{dormir} c'est du \drug{Xanax} 
    \translation{\drug{Xanax} is the only thing I was given to help me \function{sleep}}}
    % {\drugStop{Opripramol} \trigger{hatte} ich ja nur \timexDur{zwei Abende lang}~zum \function{Schlafen} je \measure{eine halbe} genommen, also nur eine \measure{winzige Dosis}. 
    % \translation{I \trigger{had} only taken \measure{half a dose} of \drugStop{Opripramol} \timexDur{for two evenings} for \function{sleeping}, so only a \measure{tiny dose}.}}
    {\drugStop{Opripramol} \trigger{nahm} ich nur \timexDur{zwei Abende lang}~zum \function{Schlafen} (...). 
    \translation{I only \trigger{took} \drugStop{Opripramol} \timexDur{for two evenings}~for \function{sleeping} (...)}}
    {\madeup \function{白血球}は問題なかった。\translation{no problem with \function{leukocyte}}} % no problem with leukocyte


\subsubsection{Negation of Functions}\label{sec:neg_func}
% However, as stated above, adverse physiological processes are annotated as \dis, and therefore, only \emph{working} body functions are annotated as \texttt{function}.
For \texttt{function}, we add the negation attribute when the function goes wrong.
It is not limited to absence or stopping (e.g,. ``\functionNeg{appetite} disappears''), but also includes abnormality, i.e., when the original body processes changes in any way (e.g., ``\functionNeg{WBC} is decreasing'').
Note that \textit{continuous (noun/verbal) phrases} of malfunctions should be annotated not as negated \texttt{function}, but as \dis: e.g., ``I have \disorder{no appetite}'' and ``\disorder{decreased WBC} was observed'' because malfunctions can be regarded as \dis. See also Section \ref{sec:dis_vs_func}.

% \bigskip (this commented paragraph is just a note, could be deleted)
% {
% \color{red}{
% Especially Japanese often texts need ``negated functions'' because a style ``[function noun] is [negative adj/verbal-noun]'' often appears.
% Without ``negated function'', we would have to annotate the \textbf{whole sentences} as \dis.
% 
% I have no appetite: 
% \begin{tabular}{c|c|c|c}
%     まったく & 食欲 & が & ない \\
%     at all & appetite & is & nothing \\
%     (adv) & (noun) & (verb) & (adj) \\
% \end{tabular}\bigskip

% decreased white blood cell count: 
% \begin{tabular}{c|c|c}
%     白血球数 & が & 低下 \\
%     \# of WBC & is & fall \\
%     (noun) & (verb) & (verbal-noun) \\
% \end{tabular}
% }}

\subsection{Anatomy}\label{ent_anatomy}

With \anatomy{\texttt{anatomy}}, we annotate all organs or anatomical parts.
We usually do not annotate smaller parts such as partial tissues and blood cells as \texttt{anatomy}, but as \texttt{function}. 
However, if a sentence describes a disorder found in a cell, the cell could be an \texttt{anatomy} entity.


\examples{ex:anatomy}
    {Had numerous odd \disorder{aches}, especially in the \anatomy{leg area}.}
    {(\ldots) mais elle a fait une \disorder{réaction} au \anatomy{pancréas}, (\ldots). \translation{but her \anatomy{pancreas} was \disorder{affected},}}
    {Ich besitze nur noch eine \anatomy{Niere}. \translation{I only have one \anatomy{kidney} left.}}
    {\madeup \anatomy{お腹}が\disorder{刺すように痛い} \translation{I feel a \disorder{stabbing pain} in my \anatomy{stomach}}} % I feel a stabbing pain in my stomach (範囲は全体を対象とする:e.g. お腹から頭にかけて)


\noindent \texttt{Anatomy} entities are \textit{not annotated} when within a larger entity such as within a \dis or \texttt{test}.
Therefore, we prioritize the annotation of \dis and \texttt{test}.

\examples{ex:anatomy_embedded}
    {\disorder{Pain} under \anatomy{ribs} , \disorder{restless legs} .}
    { J'ai beaucoup de \disorder{douleurs} \timexFreq{au quotidien} : \disorder{maux de tête},\\ \disorder{douleurs musculaires}  \translation{I am in \disorder{pain} \timexFreq{everyday} : \disorder{headaches}, \disorder{sore muscles}}}
    {\test{Magen/Darmspiegelung}~, weil ich immer \disorder{Magenschmerzen} hatte. \translation{\test{Gastrointestinal endoscopy} because I always had \disorder{stomach pain}}}
    {\madeup \disorder{肺がん}と宣告されてしまった \translation{I was diagnosed with \disorder{lung cancer}.}}


\subsection{Test}\label{ent_test}

With \texttt{test}, we mark all medical tests, interviews, examinations or any other procedure that produces a result to be used in medical diagnoses.

\examples{ex:test}
    {\test{Blood test} was \measure{normal}.}
    % {des \disorder{douleurs} dans mes \anatomy{pouces} surtout celui de gauche, 
    {je passe une \test{radio} des \anatomy{2 mains} \timexPIT{mercredi matin}, avant ma \drug{perf}, (...) \translation{I'm having a \test{CT} for \anatomy{both hands} on \timexPIT{Wednesday morning}, before my \drug{IV}}}
    {Ich habe ganz \timexPIT{zu Beginn}~ der \function{WJ} auch mal \test{Speicheltests} machen lassen, mit zum Teil abstrusen Werten. \translation{\timexPIT{At the very beginning}~ of the \function{MP}, I also had \test{saliva tests} done, some of which showed abstruse values.}}
    {\madeup \test{血液検査}の結果が怖いです \translation{I'm afraid of the \test{blood test} results.}} % I'm afraid of the blood test results.
    
    


\subsection{Opinion} \label{ent_eval}

This denotes the \textit{personal} evaluation or opinion of a medication (\dr), health state (\dis) or biological process (\texttt{function}).
This entity could be a detailed description of a mental function or disorder (e.g., ``My \function{feeling} is \opPos{stable}'').
% ``the \disorder{side effect} was \opNeg{so terrible}''
% I'm so sad that I lost my appetite.
Usually, these assessments are rather colloquial, and it is difficult to find an appropriate span.
Therefore, we once again follow the principle of taking the shortest span possible.

The opinion entity always comes with a sentiment attribute: either \texttt{positive}, \texttt{negative}, or \texttt{neutral}.
A \texttt{positive} assessment is often associated with an improvement of a disease or with a good experience of the patient with a certain medication.
We do not assign \textit{negated} (the \textit{negation} attribute) to the corresponding \texttt{disorder}.
We annotate \textit{all} opinions to make it easier for the annotators and keep the annotation consistent with the other entities. 
Patient-related opinions are then explicitly expressed by relations.

% Painkiller[DRUG_stopped] –CAUSED→ Pain (TREATMENT_FOR also applies here)

\examples{ex:eval}
    % {*I also \trigger{had} \drugStop{antibiotics} but that \opNeut{didn't change anything}.\\ Then, after \trigger{starting} \drugStart{Trisequen's hormone pills}, just \timexDur{1 week later}, I was \opPos{like in 7th heaven}.}
    % 
    {\translated This \drug{drug} is a \opNeg{nightmare} and \opNeg{should be discontinued}.}
    {l'\drug{entyvio} c 'est pas miracle mais bon \opPos{ça aide}. \translation{\drug{entyvio} is no miracle, but well, \opPos{it does help}.}}
    {Ich nehme doch jetzt \timexDur{12 Wochen}~\trigger{kein} \drugStop{Lyrica} \trigger{mehr} und \trigger{dafür} \measure{50mg} \drugStart{Opi} und das \opPos{lief auch richtig gut}. \translation{I \trigger{did not take} \drug{Lyrica} for \timexDur{12 weeks}~ and \trigger{instead} took \measure{50mg} of \drug{Opi} and that \opPos{went really well}, too.}}
    {\translated \drug{アモキサン}\opPos{効いてる}おかげか\function{気持ち}は\opPos{少し楽になった} \translation{My \function{feeling} becomes \opPos{a little better} now that the \drug{amoxan} \opPos{is working}.}} % I'm feeling a little better now that the amoxan is working.
    
\paragraph{Emojis / Emoticons} We include emojis or emoticons if they explicitly encode an opinion.

% get code points here: https://ctan.math.washington.edu/tex-archive/macros/luatex/latex/emoji/emoji-doc.pdf
% \examples{emoji}
%     {**This \drug{drug} was \opNeg{{\NotoEmoji \symbol{"1F922}}}}
%     {**Ces \opNeg{{\NotoEmoji \symbol{"1F92C}}} médecins ne savent même pas ce qu'ils prescrivent!}
%     {**Der Arzt hat mir endlich wieder mein \drug{AD} verschrieben \opPos{{\NotoEmoji \symbol{"1F64F}}}}
%     {**}


\subsubsection{Attribute: Sentiment}\label{sec:sent}


When patients describe their current state of health or when they assess the consequences of a medication they took, they often use emotional words.
Therefore, we add, if applicable, the sentiment markers \texttt{positive}, \texttt{negative}, or \texttt{neutral} to the phrases annotated with \texttt{opinion}.

\examples{ex:sentiment}
    {I really made a difference, \opPos{improvement of quality of life}.}
   % {Bonjour \user{xxxxxxxx} j'ai \opPos{très bien toléré} \drug{entyvio} \disorderNeg{aucun effet indésirable}.}
    {\opNeg{C'est nul} le \drug{Xanax} \opNeg{ça marche même pas} \translation{\drug{Xanax} \opNeg{sucks}, \opNeg{it doesn't even work}]}}
    {Dann, nachdem ich mit \drugStart{Triseqenz Hormontabl.} \trigger{angefangen} habe, nur \timexDur{1 Woche später}~war ich \opPos{wie im 7. Himmel}. \translation{Then, after I \trigger{started} \drugStart{Triseqent hormon pills}, just \timexPIT{one week later}, I was \opPos{on cloud nine}.}}
    {\drug{レクサプロ}は\disorder{強迫}には\opNeg{効かなかったです} \translation{\drug{Lexapro} \opNeg{didn't help} with \disorder{obsessive thoughts}.}} 

\subsection{Measure} \label{ent_measure}

With \measure{\texttt{measure}}, we mark clinically relevant measurements, such as drug dosages and test results.
The expression is typically a numerical value that accompanies a measurement unit.

\examples{ex:measure}
    {Started \timexPIT{2 years ago}~ with \measure{10 mg} then \timexDur{6 mos later}\\ \doctor{doc} upped to \measure{20}.}
    % {\ldots, bon ça fait beaucoup, \timexDur{depuis 1 an et demi} ,perfusions d entyvio tous les mois. 
    {\drug{pentasa}, \drug{cortancyl}  (\measure{10mg}), (\ldots) 
    \translation{\drug{pentasa}, \drug{cortancyl}  (\measure{10mg}), \ldots}}
    {Das \drug{Utrogest} sind \route{Weichkapseln} mit \measure{100 mg} \drug{naturid.~Progesteron} und sie sind verschreibungspflichtig. 
    \translation{\drug{Utrogest} are \route{soft capsules} with \measure{100 mg} of \drug{nature-id. progesterone} and they are prescription-only.}}
    {\test{生検}の結果\function{Ki67}が\measure{46%}だったため、通院にて治療中です 
    \translation{The \test{biopsy} results showed a \function{Ki67} of \measure{46\%}, and I am in the hospital for treatment. }} % The biopsy results showed a Ki67 of 46\%, and I am in the hospital for treatment. 

In some cases, it might happen that it is difficult to distinguish between a phrase being a \texttt{opinion} or a \texttt{measure} when they occur together with a \texttt{test} (see, for example, \Cref{sec:diff_cases_en}, Ambiguity). 
In these cases, we prioritize \texttt{measure.}

\subsubsection*{Temporal Measurements}

Note that temporal measurements (e.g. ``5 times per month'') should be annotated as \texttt{time} entities (see Section \ref{ent_timex}) \textit{unless} they indicate the amount/dosage of a drug:
If we can relate the expression with the \texttt{dosage} relation to a \dr, then it is probably a \texttt{measure}.

\begin{enumerate}

\item \madeup I took \measure{\textcolor{white}{three days' worth}} of \drug{medicines} at once \timexDate{today}~.

\item \madeup My mom should take this \drug{pill} \measure{\textcolor{white}{three times a day}}.
\end{enumerate}

\subsection{Time} \label{ent_timex}

As temporal markers (\timex{\texttt{time}}~), we define all mentions of \textit{frequencies}, \textit{durations}, \textit{dates}, or \textit{relative points in time}. 
For a more narrow description, we use those characteristics (frequency, duration, \ldots) as attributes.
Also, we include prepositions (e.g., ``in'', ``from'', ``before'', and ``since'') in the entity since these carry relevant information specifying the semantics of the expression.
If there is no suitable attribute, we leave it blank (i.e., we do not add an attribute).

Time expressions include e.g., ``night'', ``afternoon'', ``for one week'' (mostly \texttt{duration}), but also expressions like ``last Monday'', ``in two weeks'' (\texttt{relative points in time}), ``every morning'', ``after lunch'' (\texttt{frequency}) or ``11.07.2022`` (\texttt{date}).

\examples{ex:time}
    {Instead of taking the \drugDec{pill} \measure{2X daily}, as prescribed I take it (\ldots) usually \measure{about 3-4 times per week}, but often skipping \timexFreq{a week or two at a time}.}
    {Bonjour, Je suis sous \drug{entyvio} \timexDur{depuis juillet 2019}~et ce traitement\\ \opPos{fonctionne fort bien} au niveau de la \disorder{colite}. \translation{Hi, I've been taking \drug{entyvio} \timexDur{since july 2019}~and the drug\\ \opPos{has been working very well} for my \disorder{colitis}.} }
    {Hallo liebe $<$user$>$, \timexPIT{vor 2 Monaten}~habe ich meine \doctor{Frauenärztin} gefragt, da ich zwischendurch einen \test{Hormontest} wollte. \translation{Hello dear $<$user$>$, \timexPIT{2 months ago} I asked my \doctor{gynecologist} because I wanted a \test{hormone test} in the meantime.}}
    {\timexDate{3年前の2月ごろに}\disorder{乳がん}が発覚 \translation{\disorder{Breast cancer} was discovered \timexDate{around February 3 years ago}.}} % Breast cancer was discovered around February 3 years ago.


\subsubsection{Attribute: Temporality}\label{sec:timex_attributes}

The following attributes of time expressions are considered in our annotations:
\texttt{frequency}, \texttt{duration}, \texttt{date} and \texttt{relative point in time}. 
The attributes should help to specify the annotated time expressions in more detail. See examples in \Cref{ent_timex}.
If none of the attributes fits, we do not add one.

\subsection{Route} \label{ent_route}

\route{\texttt{Route}} annotates the means of medication intake, e.g., via pills or via injection.
We, thus, annotate verbs indicating medication intake, such as ``drink'' or ``inject'', but exclude too general verbs (e.g. ``have'' and ``take'').
If a mention like ``pill'' refers to a drug (and not to the means of intake), it should be annotated as \dr.
\examples{ex:route}
    {It has not eliminated the need for \route{oral} \drug{pain meds} in all situatations but \opPos{has helped}.}
    {(\ldots), \timexDur{depuis 1 an et demi}, \route{perfusion} d \drug{entyvio} \timexFreq{tous les mois}. (\ldots) 
    \translation{\timexDur{for the past 18 months}, I get \drug{entyvio} \route{IV}  \timexFreq{every month}.}}
    {Ich habe das leider nur mit \route{Tabletten} gemacht (mir wollte keiner meiner \doctor{Ärzte} eine \route{Infusion} verschreiben). 
    \translation{Unfortunately, I only did this with \route{pills} (none of my \doctor{doctors} would prescribe me an \route{infusion}).}}
    {\translated 現在\route{内服}しているのは\drug{レトロゾール}です 
    \translation{I am currently taking \drug{letrozole} \route{orally}.}} 


\subsection{Doctor} \label{ent_doctor}
We annotate medical job descriptions, such as \textit{physician}, \textit{dentist}, \textit{nurse}, \textit{(psycho) therapist}, etc., with \doctor{\texttt{doctor}}.
This entity label is for job descriptions, \textit{not} for the given names of doctors.

\examples{ex:doctor}
    {I took \drugInc{lipitor} \measure{10mg} \timexDur{for only about two months},then \doctor{cardiologist} \trigger{increased} it to \measure{20mg}.}
    {( peut etre de \disorder{l'arthrose} dixit mon \doctor{docteur traitant}) (\ldots) \translation{(could be \disorder{arthritis} according to my \doctor{GP}) (\ldots)}}
    {(\ldots), hatte ich \timexPIT{gestern}~einen Termin bei meinem \doctor{Internisten}. \translation{(\ldots), I had an appointment with my \doctor{internist} \timexPIT{yesterday}}}
    {\madeup 病院の\doctor{先生}がこの薬を\route{飲め}って言うんです。でも信用できません \translation{The \doctor{doctor} at the hospital says I should \route{drink} this medicine. But I don't trust him.}} 


\subsection{Entities for De-Identification}\label{sec:deident}

To make sure personal information like \user{\texttt{user names}}, \link{\texttt{URLs}}, \info{\texttt{e-mail}} addresses, etc., do not end up in the final corpus, even if we pre-processed the data, we also mark these information with the respective labels to later apply a final post-processing.
Replacing those entities with their corresponding marker, e.g., $<$user$>$, should not change the content of the document.

\subsubsection*{User} \label{ent_user}

Since social media users use creative names that are not necessarily easy to find using pre-processing tools, we mark ``left-over'' user names to de-identify them afterward if necessary.
The examples below were randomly generated from scratch and not taken from real existing data.

\examples{ex:user}
    {\madeup Dear \user{Leopard Footballer}, thank you for your message.}
    {\madeup Bonjour \user{Mûre Vive} pour ma part ils \opNeg{n'ont pas été efficace} (\ldots) \translation{Hi \user{Mûre Vive}, for me they were \opNeg{not effective}}}
    {\madeup Liebe \user{Broccoli Klarinette} danke für deinen Bericht. \translation{Dear \user{Broccoli Klarinette}, thanks for your report.}}
    {\madeup \user{ゆーりん}さん、ありがとうございます!\translation{Thank you, \user{Yurin}!}} 

\subsubsection*{URL} \label{ent_url}
In case there are any URLs or e-mail addresses missed when de-identifying the data, we mark them to be removed later.

\examples{ex:url}
    {\madeup Go to \link{www.xxxxx.xx.xx} for more info.}
    {\madeup Si vous souhaitez discuter plus avant: \link{www.xxx xxxx xxx xxx}. \translation{If you wish to discuss further: \link{www.xxx xxxx xxx xxx}.}}
    {\madeup Wenn du möchtest, geh doch mal auf die Seite \link{www.xxx xxxx xxx xxx}. \translation{If you like, take a look at this website: \link{www.xxx xxxx xxx xxx}.}}
    {\madeup ここに詳しく載ってます \link{www.xxx.xx}} 

\subsubsection*{Personal information} \label{ent_personal}

With this marker, we annotate all other personal information that needs to be removed before data publication (jobs, city, doctor/hospital' names ..).
This kind of annotation is only necessary if the previous (automatic) de-identification failed.
The examples below were randomly generated.

\examples{ex:info}
    {\madeup My name is \info{Emanuel Streich} and I am \info{16 years old}.}
    {\madeup J'ai appris que j'avais \disorder{crohn} \timexPIT{il y a cinq ans}. J’avais \info{20 ans}. \translation{I was diagnosed with \disorder{crohn} \timexPIT{five years ago}. I was \info{20 years old}.}}
    {\madeup Ich wohne in \info{Musterhausen}, das kennst du sicher. \translation{I live in \info{Musterhausen}, you probably know it.}}
    {\madeup \info{伊藤晴美}、\info{大学2年生}です。\translation{I am 
\info{Harumi Ito}, a \info{second-year college student}.}} 



\subsection{Other} \label{ent_other}

With \other{other}, we annotate all remaining entities that refer to a kind of treatment or medical event (pregnancy, wound, etc.), but for which we do not have a category.
For example, since we do not have an entity for treatments that are not drugs (e.g., \textit{cognitive behavioral therapy}), we would annotate this term as \other{cognitive behavioral therapy}.
Additional examples are clinical tools (e.g. ``syringe'') and medical devices (e.g. ``wig'', ``dental implant'', and ``pacemaker''), as well as operations, e.g., ``aesthetic surgery''.
\examples{ex:other}
    {Just \other{physical therapy} and \drug{pain medication}.}
    {Mon \doctor{médecin généraliste}  m'a recommandé la \other{psychothérapie} et le \drug{Femiloges}.
    \translation{My \doctor{GP} recommended \other{psychotherapy} and \drug{Femiloges}.}}
    {Ohne BH fallen die \other{Implantate} runter. 
    \translation{Without bra the \other{implants} fall off.}}
    {\translated \drug{抗がん剤}の\disorder{副作用}で\anatomy{髪}が\disorder{抜け}て、\other{ウィッグ}を着用していました。
    \translation{Wore a \other{wig} because of \disorder{hair loss} due to \disorder{side effects} of \drug{anti-cancer drugs}}} % Wore a wig because of hair loss due to side effects of anti-cancer drugs

\clearpage
\section{Relations}\label{sec:relations}

In the following, we will give details on how we annotate the relations between the entities.
All relations were defined to be language-independent and to be used also with little linguistic knowledge.
As an exception, \other{other} can take and accept all relations for practical reasons.
We further take into account the following:
\begin{itemize}
    \item The interpreted intention of the author if the annotator(s) judges it to be clearly identifiable.
    \item Common world knowledge (e.g., the action-reaction principle) and common sense (e.g., a broken leg can not be caused by brushing one's teeth even if the text author insisted that it was).
    \item A possible contextual cue from metadata (e.g., the metadata might contain a drug name, which is not necessarily mentioned in the text itself).
\end{itemize}
We do \textit{not} annotate relations if one of the following points apply:

\begin{itemize}
    \item The entities described in \Cref{sec:entities} are not concerned, i.e., again, we follow the perspective of the patient.
    \item The document at hand is hypothetical or speculative (from the patient's perspective).
    \item The document is formulated as a question.
\end{itemize}
For better visualization, we highlight only the relations and entities in question.
Also, we cut off parts of the original documents to keep every example on one line.
In \Cref{tab:relations}, we provide an overview of all relations and their arguments.


\begin{table}[H]\setlength{\tabcolsep}{3pt}
\centering
\begin{tabular}{@{}llp{16em}l@{}}
\toprule
\textbf{section} & \textbf{relation} & \textbf{argument 1} & \textbf{argument 2}\\ \midrule
\ref{rel_caused} & \texttt{caused} & \{\dr, \dis\} & \{\dis, \texttt{function}\} \\
\ref{rel_treat} & \texttt{treatment\_for} & \{\dr\} & \{\dis, \texttt{function}\} \\
\ref{rel_dosage} & \texttt{has\_dosage} & \{\dr\} & \{\texttt{measure}\} \\
\ref{rel_experienced} & \texttt{experienced\_in} & \{\dis\} & \{\texttt{anatomy}\} \\
\ref{rel_examined} & \texttt{examined\_with} & \{\dis, \texttt{anatomy}, \texttt{function}\} & \{\texttt{test}\} \\
\ref{rel_resulted} & \texttt{has\_result} & \{\texttt{test}\} & \{\texttt{measure}, \texttt{disorder}, \texttt{function}\} \\
\ref{rel_refers} & \texttt{refers\_to} & \{\dis\} & \{\dis, \texttt{function}$^{negated}$\} \\
\ref{rel_refers} & \texttt{refers\_to} & \{\dr\} & \{\dr\} \\
\ref{rel_refers} & \texttt{refers\_to} & \{\texttt{anatomy}\} & \{\texttt{anatomy}\} \\
\ref{rel_refers} & \texttt{refers\_to} & \{\texttt{function}\} & \{\texttt{function}\} \\
\ref{rel_interacted} & \texttt{interacted\_with} & \{\dr\} & \{\dr\} \\
\ref{rel_change} & \texttt{signals\_change\_of} & \{\texttt{change\_trigger}\} & \{\dr\} \\
\ref{rel_time} & \texttt{has\_time} & \{\dr, \dis\} & \{\texttt{time}\} \\
\ref{rel_means} & \texttt{has\_route} & \{\dr\} & \{\texttt{route}\} \\
\ref{rel_eval} & \texttt{is\_opinion\_about} & \{\texttt{opinion}\} & \{\dr, \dis, \texttt{function}\} \\
\ref{rel_misc}& \texttt{misc} & \{\texttt{ANY}\} & \{\texttt{ANY}\} \\

& \textit{not tracked} & \texttt{doctor, user, URL, personal info, other} & \\
\bottomrule
\end{tabular}
\caption{Overview of available relations and the entities they associate.}
\label{tab:relations}
\end{table}




\subsection{caused}\label{rel_caused}


We only annotate a \texttt{caused} relation when the entities \dr, \dis, or \texttt{function} are concerned. 
For this, we take into account the following:
\begin{itemize}
    % \item The interpreted intention of the author if the annotator(s) judges it to be clearly identifiable.
    % \item Common world knowledge (e.g., the action -- reaction principle) and common sense (e.g., a broken leg can not be caused by brushing one's teeth even if the text author insisted that it was).
    % \item A possible contextual cue from metadata (e.g., the metadata might contain the drug name, which is not necessarily mentioned in the text itself).
    \item Explicit formulation of a \cause--\conseq relation, i.e., supported by linguistic markers like:
    \begin{itemize}
        \item Explicit discourse inter-clause/sentence articulators: \textit{because}, \textit{so}, \textit{then}, \ldots (including conjunction coordination hinting at  causal entailment like ``and'', ``then''). %      <cause clause> then <consequence clause>
        \item Conditional constructions: ``when \cause , \conseq'',  \\ ``\cause and \conseq'', \ldots
        \item Restrictive formulations, e.g., ``only when \cause , \conseq''.
    \end{itemize}
    \item Lexical semantics of nouns or verbs, e.g.,:
    \begin{itemize}
        \item \cause  provokes \conseq
        \item \conseq is the consequence of \cause
        \item \cause was followed by \conseq
        \item \cause entailed \conseq
        \item \conseq is correlated with \cause
        \item \cause is correlated with \conseq and \cause is preceding \conseq
        \item \cause is probably linked with \conseq and \cause is preceding \conseq
    \end{itemize}
    \item \cause \conseq relations are reported by the author of the text (first person) or attributed to another person by the author of the text (second/third person, like parents or siblings). This does not apply when one user provides advice to other users.
    \item Being part of a juxtaposition of linguistic element/clauses/sentences (but only if supported by the context, either co-text and/or metadata elements):
    \begin{itemize}
        \item \cause \conseq
        \item \cause, \conseq
        \item \cause. \conseq.
    \end{itemize}
    \item Being part of a successive temporal relation:
    \begin{itemize}
        \item \conseq after \cause
        \item before \conseq \cause
        \item every time \cause \conseq
    \end{itemize}
    \item Juxtaposition of \cause--\conseq clause and change of factuality/belief/veracity/opinion:
    \begin{itemize}
        \item \cause--\conseq belief clause  $<$veracity expression about previous clause$>$
        \item ``I though I would not have \conseq because of \cause. I was wrong!''
        \item ``I had never never thought that \cause would yield \conseq
        \item ``I would not have expected that \cause would yield \conseq
    \end{itemize}


 
\end{itemize}





\examples{ex:caused}
    {\drawRelations{
        The \& \drug{pill} \& does \& its \& job \& , \& by \& I \& \disorder{constantly feel sick} \& to my stomach}
        {
        \causedRel{2}{9}
        }
    }
    {\drawRelations{
        Elle a pris \& de l'\drugStop{infliximab} \& et (\ldots) elle est devenue \& \disorder{toute rouge} \& et \& \disorder{nauséeuse}}
        {
        \causedRel{2}{4} 
        \causedRel{2}{6}
        }
    }
    {\drawRelations{
        Hatte (\ldots) auch ein \& \drug{AD} \& ausprobiert. Bei mir hat es den \& \disorder{TSH hochgetrieben}!}
        {
        \causedRel{2}{4}
        }
    }
    {\drawRelations{
        % \drug{レクサプロ} \& \trigger{飲み始めてから} \& \disorder{吐き気} \& はするし \& \disorder{食欲無い} \& し 
        \drug{レクサプロ} \& 飲み始めてから \& \disorder{吐き気} \& はするし \& \functionNeg{食欲} \& 無いし 
        }{
        \causedRel{1}{3}
        % \changeRel{2}{1}
        \causedRel{1}{5}
        }
    } 

%Ensuite elle a pris infliximab mais au bout de 3 perfusion elle a développé des anticorps et grosse réaction allergique avec tachycardie importante et malaise, elle devient tout rouge et nausée, donc arrêt aussi.
% Hatte in meiner Verzweiflung ja auch ein AD ausprobiert. Bei mir hat es den TSH voll hochgetrieben!


\subsubsection{Special Case: Effects of Stopping Medication Intake}

% Officially, drug withdrawal is an adverse drug event. 
% Therefore, 
We include ``drug withdrawal'' (i.e., symptoms often caused by discontinuation of an addictive drug) and draw a \texttt{caused} relation between the \dr and the withdrawal symptoms (\dis).

Stopping a medication might, however, also bring back the original symptoms or disorder. To ease the annotation burden and make the annotation consistent, we always draw a \texttt{caused} relation between a \textit{stopped} \dr and the resulting \dis, if there is any. 
Note that for the latter, there is most probably also a \texttt{treatment\_for} relation.



\subsection{treatment\_for}\label{rel_treat}

This relation connects a \dr and the targeted \dis, describing the medication that was used to treat the patient's symptoms.

\examples{ex:treatment}
    {\drawRelations{
        I have \& \disorder{insomnia} \&. My doctor prescribed this \& \drug{medication} \& to me for better sleep.
        }{
        \treatmentRel{4}{2}
        }
    }
    {\drawRelations{
    J'ai ressorti le \& \drug{Lasea} \& de l'armoire parce que quand je l'avais pris, \& \disorder{l'anxiété} \& \\n'était pas si forte.}{
        \treatmentRel{2}{4}    
        }
    }
    {\drawRelations{
        Habe vor 4-5 jahren \& \drug{isotretinoin} \& wegen \& \disorder{starker akne} \& eingenommen
        }{
        \treatmentRel{2}{4}
        }
    }
    {\drawRelations{
        % \timexDate{先週から} \& \disorder{乳がん} \& の \& \drug{ホルモン剤} \& を \& \route{服用} \& \trigger{開始しました}
        先週から \& \disorder{乳がん} \& の \& \drug{ホルモン剤} \& を \& \route{服用} \& \trigger{開始しました}
        }{
        % \timeRel{1}{4}
        \treatmentRel{4}{2}
        % \trigger{7}{4}
        }
    } 


\subsection{has\_dosage}\label{rel_dosage}

\texttt{has\_dosage} relates a measurement, e.g.~the number of pills (\texttt{measure}), with a medication name (\dr).

\examples{ex:dosage}
    {\drawRelations{
        I took \& \drugInc{lipitor} \& \measure{10mg} \& \timexDur{for only about two months}, (\ldots)
    % then \doctor{cardiologist} \trigger{increased} it to \& \measure{20mg}.}
        }{
        \dosageRel{2}{3}
        % \dosageRel{2}{5}
        }
    }
    {\drawRelations{
        (...) \trigger{le premier essai} de \& \drugStart{l'Utrogest} \& \measure{200} \& a été un \opNeg{échec total}.}{
            \dosageRel{2}{3}
        }
    }
    {\drawRelations{\drug{Utrogest} sind verschreibungspflichtige Kapseln mit \& \measure{100 mg} \&  \& \drug{naturid.~Progesteron} \& .}{\dosageRel{4}{2}}}
    {\drawRelations{
        \drug{エビリファイ} \& \timexDate{初日} \& ですが \& \measure{3mg} \& からなので \& \opNeut{効き目はよくわかりません}
        }{
        % \timeRel{2}{1}
        \dosageRel{3}{1}
        }
    } 
    

\subsection{experienced\_in}\label{rel_experienced}

This relation associates a \dis with the location it is felt/experienced in, i.e., part of the body (\texttt{anatomy}).


    % {\drawRelations{
    %     Hatte (\ldots) auch ein \& \drug{AD} \& ausprobiert. Bei mir hat es den \& \disorder{TSH hochgetrieben}!}
    %     {
    %     \causedRel{2}{4}
    %     }
    % }

\examples{ex:experienced}
    {\drawRelations{
        \disorder{Pain} \& under \& \anatomy{ribs} \&, \disorder{restless legs} .}
        {
        \experiencedRel{1}{3}
        }
    }
    {\drawRelations{j'ai eu un gros souci de \& \disorder{désydrose} \& aux \& \anatomy{pieds}}{\experiencedRel{2}{4}}}
    {\drawRelations{\madeup Ich hatte einen \& \disorder{seltsamen Schmerz} \& im \& \anatomy{Bein} \& .}{\experiencedRel{2}{4}}}
    {\drawRelations{
        \anatomy{顔} \& と \& \anatomy{足} \& が \& \disorder{凄くむくむ}
        }{
        \experiencedRel{5}{1}
        \experiencedRel{5}{3}
        }
    } 

\subsection{examined\_with}\label{rel_examined}

This relation associates a \dis or a \texttt{function} with a \texttt{test}: the \texttt{test} is used to examine the \dis.
\texttt{anatomy} entities can be examined with a \texttt{test}, too.

\examples{ex:examined}
    {\translated \drawRelations{I'm having \& \anatomy{both hands} \& \test{x-rayed} \& \timexPIT{on Wednesday morning}~, (\ldots)}{
        \examinedRel{2}{3}
        }
    }
    {\drawRelations{je passe une \& \test{radio} \& des 2 \& \anatomy{mains} \& \timexPIT{mercredi matin}~, (\ldots)}{
        \examinedRel{4}{2}
        }
    }
    {\drawRelations{
        \madeup Meine \& \disorder{Schmerzen} \& wurden dann mit einer \& \test{Magenspiegelung} \& untersucht.
        }{
        \examinedRel{2}{4}
        }
    }
    {\drawRelations{
        % \timexDate{先月} \& の \& \test{検査} \& で \& \anatomy{肝臓} \& の \& \disorder{転移} \& を言われてしまいました
        先月 \& の \& \test{検査} \& で \& \anatomy{肝臓} \& の \& 転移 \& を言われてしまいました
        }{
        % \timeRel{1}{3}
        \examinedRel{3}{5}
        % \experiencedRel{7}{5}
        }
    }    
    
    
\subsection{has\_result}\label{rel_resulted}

Following the line of thought from above, the \texttt{has\_result} relation then relates the \texttt{test} with its result, usually a \texttt{measure}, in rare cases also a \dis or a \texttt{function}.
This relation should not be used for \texttt{caused} scenarios.

\examples{ex:resulted}
    {    
    \drawRelations{\test{Blood test} \& was \& \measure{normal}.}
    {
    \resultedRel{1}{3}
    }}
    {\drawRelations{J'ai fait un \& \test{test de salive hormonale} \& (\ldots) Ici, indication plus claire de  \& \function{ménopause}.}{
        \resultedRel{2}{4}
        }
    }
    {\drawRelations{
        \madeup Die \& \test{Spiegelung} \& zeigte \& \measure{keinerlei Ergebnisse}.
        }{
        \resultedRel{2}{4}
        }
    }
    {\drawRelations{
        \madeup 先月 \& の \& \test{検査} \& では \& 白血球 \& は \& \measure{5000} \& でした
        }{
        % \timeRel{1}{3}
        % \examinedRel{3}{5}
        \resultedRel{3}{7}
        }
    } 
    
Note that the mention of a \texttt{function} can implicitly refer to a test, resulting in a measure.
Therefore, we also allow \texttt{has\_result} relations between functions and measures. 
    

\subsection{interacted\_with}\label{rel_interacted}

In some cases, a \dr--\dr interactions result in a \dis. 
The relation \texttt{interacted\_with} is defined for exactly this purpose.

\examples{ex:interacted}
    {\drawRelations{\translated I've already tried \& \drug{Laiff} \& \measure{900}, it reduces the effect of my \& \drug{progesterone}.}{
        \interactedRel{2}{4}
    }}
    {\drawRelations{J'ai déjà essayé le \& \drug{Laiff} \& \measure{900}, il réduit l'effet de ma \& \drug{progestérone}.}{
        \interactedRel{2}{4}
    }}
    {\drawRelations{
        \madeup Die Kombi aus \& \drug{Grapefruitsaft} \& und \& \drug{Aspirin} \& hat meine \disorder{Schmerzen} nur verschlimmert.
        }{
        \interactedRel{2}{4}
        }
    }
    {\drawRelations{
        \translated \& \drug{レクサプロ} \& と \& \drug{ビール} \& \route{飲んだら} \& \disorder{嘔吐} \& した
        }{
        \interactedRel{2}{4}
        % \causedRel{2}{6}
        % \causedRel{4}{6}
        }
    }

\subsection{signals\_change\_of}\label{rel_change}

For connecting the \texttt{change triggers} with the \dr that is affected, we use the \texttt{signals\_change\_of} relation.

\examples{ex:re_change}
    {\drawRelations{\madeup Last week, I \& \trigger{started} \& the \& \drug{medication therapy}. }{
        \changeRel{2}{4}
    }}
    {\drawRelations{
        depuis que j'ai \& \colorbox{celadon}{arrêté$_{\texttt{trigger}}$} \& \drugStop{deroxat} \& \colorbox{celadon}{pour$_{\texttt{trigger}}$} \& \drugStart{effexor} \& j'ai perdu 3 kilos (\ldots)}
        {
        \changeRel{2}{3}
        \changeRel{2}{5}
        }
    }
    {\drawRelations{
        \madeup Ich habe mit \& \drugStart{Triseqenz Hormontabl.} \& \trigger{angefangen} \& und fühle mich super.
        }{
        \changeRel{2}{3}
        }
    }
    {\drawRelations{
        \drug{リフレックス} \& から \& \drug{レクサプロ} \& に \& \trigger{変えた}     
        }{
        \changeRel{5}{1}
        % \causedRel{1}{6} % \& \disorderNeg{ふらつき} \& が \& 治まった
        }
    } 

\subsection{has\_time}\label{rel_time}

The relation time connects a \texttt{time} expression with all possible entity types except other \texttt{time} expressions.
% either a \texttt{drug} or a \texttt{disorder}.

\examples{ex:re_time}
    {\drawRelations{
    \translated I have been on \& 
    \drug{entyvio} \& 
    \timexDur{since July 2019} \&
    }
    {\timeRel{2}{3}}
    }
    {\drawRelations{
    Bonjour, Je suis sous \&
    \drug{entyvio} \&
    \timexDur{depuis juillet 2019} \&
    et ce traitement (\ldots)
    }
    {\timeRel{3}{2}}
    }
     {\drawRelations{
        \timexDur{\madeup Seit einer Woche} \& nehme ich \& \drugStart{Triseqenz Hormontabl.} \&  und fühle mich super.
        }{
        \timeRel{2}{1}
        }
    }
    {\drawRelations{
        % \drug{レクサプロ} \& \trigger{飲み始めて} \& \timexDur{2週間} \& たちました
        \drug{レクサプロ} \& 飲み始めて \& \timexDur{2週間} \& たちました
        }{
        % \changeRel{2}{1}
        \timeRel{3}{1}
        }
    } 


\subsection{has\_route}\label{rel_means}
The \texttt{has\_route} relation associates a drug with how it is consumed, i.e., the means of intake.

\examples{ex:means}
    {\drawRelations{\translated I have to say right away that I cannot tolerate \& \drug{progesterone} \& \route{capsules} \& at all.”}{
        \meansRel{2}{3}
    }}
    {\drawRelations{Je dois dire tout de suite que je ne tolère pas du tout les \& \route{capsules} \& de \& \drug{progestérone}.}{
        \meansRel{2}{4}
    }}
     {\drawRelations{\madeup Ich hatte gestern leider vergessen mir das  \& \drug{Insulin} \& zu \& \route{spritzen} \& .}{\meansRel{2}{4}}}
    {\drawRelations{%
        \disorder{生理不順} \& のため \& \drug{低用量ピル} \& を \& \route{飲んでます}
        }{%
        \meansRel{3}{5}
        }%
     \translation{I \route{orally take} \drug{low-dose pills} for \disorder{irregular periods}.}
    } 



\subsection{is\_opinion\_about}\label{rel_eval}

The \texttt{is\_opinion\_about} relation connects any entity with an \texttt{opinion} entity. 
Mostly, these will be \texttt{drug} or \texttt{function} entities, along with \texttt{disorder} and \texttt{other} ones.
Note that sometimes, it is not possible to find two explicit entities to be connected. 

\examples{ex:re_eval}
    {\drawRelations{\translated I started taking the \& \drug{mirtazapine} \& again, and I \& \opPos{felt much better}.}{\evalRel{4}{2}}}
    {\drawRelations{J'ai recommencé à prendre la \& \drug{mirtazapine} \&, et \& \opPos{je me suis sentie beaucoup mieux}.}{\evalRel{4}{2}}}
    {\drawRelations{\madeup Seit ich  \& \drugStart{Triseqenz Hormontabl.} \& nehme, fühle ich mich \& \opPos{super} \& .}{\evalRel{4}{2}}}
    {\drawRelations{%
        私は \& \drug{レクサプロ} \& が \& \opPos{正解} \& 。 \& \drug{ラミクタール} \& は \& \opNeg{最低} \& だった。 
        }{%
        \evalRel{4}{2}
        \evalRel{8}{6}
        }%
        \translation{\drug{Lexapro} was \opPos{the right choice} for me. \drug{Lamictal} was \opNeg{the worst}.}
    } 

\subsection{refers\_to}\label{rel_refers}

The \texttt{refers\_to} relation represents a unidirectional link for the entities \dr, \dis, \texttt{anatomy}, and \texttt{function}, and also for the link between \dis and \texttt{function}$^{negated}$
It is used to associate similar concepts, but \textit{not necessarily} to connect the same surface forms (e.g., ``Lipitor'' and ``Lipitor''). 
Via the \texttt{refers\_to} relation, we can automatically propagate the relation to all connected targets.
The relation allows for chaining corresponding entities.

% \begin{enumerate}
%     \item \texttt{refers\_to} relations can be chained between corresponding entity mentions
%     % \item The first relation (\texttt{caused} / \texttt{treatment\_for}) is drawn to the concept the annotator thinks is most important to the understanding of the meaning of the sentences / document.
% \end{enumerate}

\examples{ex:refers}
    {\drawRelations{\translated I had \& \disorder{chills} \& and my \anatomy{whole body} was \& \disorder{shaking}}{
        \refersRel{4}{2}
    }}
    {\drawRelations{(\ldots), des \& \disorder{bourdonnements} \& dans la \anatomy{tête} \& (des \disorder{bruits} \& \anatomy{d'oreille} )\& (\ldots)}{\refersRel{4}{2}}}
    {\drawRelations{
        \madeup Ich hatte \& \disorder{Schüttelfrost} \& und \& \disorder{zitterte} \& am ganzen Körper.
        }{
        \refersRel{2}{4}
        }
    }
    {\drawRelations{
        % \timexPIT{最近} \& \drug{抗うつ剤} \& の \& \drug{レクサプロ錠} \& \measure{10mg} \& も処方になった
        最近 \& \drug{抗うつ剤} \& の \& \drug{レクサプロ錠} \& 10mg \& も処方になった
        }{
        % \timeRel{1}{2}
        \refersRel{2}{4}
        % \dosageRel{5}{4}
        }
    }


    % \item ($R_0$ is a \texttt{refers\_to} relation) and (\conseq is the target of a \texttt{caused} relation) and (\cause is the source of a \texttt{caused} relation pointing  to the source of $R_0$) 


% \textcolor{green}{
% TODO for refers\_to
% \begin{enumerate}
% % \item link all mentions sharing a concept (those with the same surface form do not need to be linked (e.g.~ Lipitor \& Lipitor)
%     \item link super/subordinate/generic concepts? 
% \end{enumerate}
% }

\subsection{misc}\label{rel_misc}

This relation is a joker similar to the \texttt{other} entity label. 
If we clearly see a worthwhile medical relation between two entities, but there is no relation defined (yet), we can use the \texttt{misc} link.

\examples{ex:misc}
    {\drawRelations{\translated (\ldots) in between a \& \other{cardioversion} \& in mid-May between the \& \other{ablations}.}{
        \miscRel{2}{4}
    }}
    {\drawRelations{J'ai passé les derniers \& \test{examens}\& -- \& \anatomy{poumons}\&, vaisseaux sanguins - tout va bien.}{
        \miscRel{2}{4}}
    }
    {\drawRelations{(\ldots), dazwischen Mitte Mai eine \& \other{Kardioversion} \& zwischen den \& \other{Ablationen}.}{
        \miscRel{2}{4}
    }}
    {\drawRelations{%
        \other{放射線治療} \& \timexFreq{3回} \& したら, \& \function{白血球} \& の値が\& \measure{半分} \& になっていました。 
        }{%
        \miscRel{1}{6}
        }%
        \translation{After \timexFreq{three rounds} of \other{radiotherapy}, the count of \function{white blood cells} was \measure{halved}.}
    } 


\section{Difficult Cases}

In the following sections, we show examples that we found difficult to annotate.
They are either ambiguous or it is in general not clear what is meant by the patient.


\subsection{Entailment}

Implicit statements entailing or implying disorders or functions are ignored for now and are a subject of future work.

\examples{ex:entailment1}
    {\madeup My eyebrows remain halfway. \textit{(meaning the other half is gone)}}
    {}
    {}
    {まつ毛は半分残りました}
    
% \examples{ex:entailment2}
%     {I only eat bread. \textit{(meaning the person cannot eat anything else)}}
%     {}
%     {}
%     {}

\subsection{Advice from other Users}

In some cases, one user repeats the symptoms of another user in their own text and gives some advice. 
Since we are annotating the data document-wise, we do not take the original patient's view into account in this case. 
Therefore, we only annotate the entities but not the relations.

\examples{advice}
    {\madeup \emph{Your} \disorder{fatigue} is certainly due to the \disorder{heavy bleeding}.}
    {}
    {\madeup Gegen die \disorder{Hitzewellen} \emph{würde ich} \drug{Johanniskraut} nehmen.}
    {}

\subsection{Distinction of Disorder, Function, and Opinion for sentiment expressions}

Sentiment words such as ``afraid'' and ``scare'' may represent signs of mental disorders in some cases (\dis), whereas they may exhibit just functional psychological reactions in yet other cases (\texttt{function}).
Also, they form opinions towards drugs (\texttt{opinion}).
We need to distinguish them according to the local context.

\begin{itemize}
    \item \drug{レクサプロ}。\disorder{副作用}が\function{怖い}。
        \translation{\drug{Lexapro}. \function{Afraid} of \disorder{side effects}.}

    \item \timexDur{今日から}\drug{漢方}のみにされるらしい、\function{こわ}  % possibly opinion?
        \translation{They are going to make me take only \drug{Chinese medicine} \timexDur{from today}, \function{scary}!}

    \item \disorder{不眠}の\disorder{症状}も\opNeg{酷い}。\disorder{辛い}。
        \translation{I also have \opNeg{terrible} \disorder{symptoms} of \disorder{insomnia}. \disorder{Painful}.}

    \item \anatomy{髪の毛}が\disorder{なくなる}ということは女性にとって、\opNeg{最も辛い}ことです。
        \translation{The \disorder{loss} of \anatomy{hair} is the \opNeg{most painful} thing for a woman.}

\end{itemize}

\subsection{Disorder expressions using general words}

\begin{itemize}
    \item \disorder{体力がガクンと落ちて}しまった
        \translation{\disorder{My physical strength has gone through the roof}.}

        ``体力 (my physical strength)'' itself does not fall into an entity, but the whole phrase means a disorder.

    \item 何をするにも\anatomy{体}が\disorder{重い}、\disorder{だるい}、\disorder{きつい}
        \translation{My \anatomy{body} feels \disorder{heavy}, \disorder{sluggish}, and \disorder{tired} in any way}

        ``重い (heavy)'' itself does not fall into an entity, but it represents a disorder in combination with ``体 (body)''.

    \item \opNeg{地獄}のように\disorder{体調が悪い}
        \translation{My \disorder{body condition goes bad} as \opNeg{hell}}

        The phrase ``体調が悪い (body condition goes bad)'' forms a disorder only when it appears as a whole.
        Although it looks like a clause, we did not annotate it with Function (``\functionNeg{体調} が悪い (\functionNeg{body condition} goes bad)'').
        This is because this particular phrase ``体調が悪い (body condition goes bad)'' is often used like an adjective ``体調悪い (bad-feeling)'', which should be labeled ``\disorder{体調悪い} (bad-feeling)''.

\end{itemize}

\subsection{English}\label{sec:diff_cases_en}

    \paragraph{Semantics} Some descriptions can be ambiguous.

\begin{flushleft}
\drawRelations{
        \disorder{pain} \& on the \& \anatomy{left lung} \&, but \& \disorderNeg{no issues} \& were found on the \& \anatomy{right}}{
        \discontRel{3}{7}
        \experiencedRel{1}{3}
        \experiencedRel{5}{7}
        
        }
\end{flushleft}
Here, it is not entirely clear whether the right \textit{lung} or right side of the entire body is meant.
        
\paragraph{Discontinuity} As mentioned before, some entity spans need to be constructed from several sub-spans to correctly capture the meaning.
This can lead to difficult constructions and makes the processing of the data more complex. 

\begin{flushleft}
\drawRelations{\translated I took \& \measure{{three days' worth}} \& of \drugInc{medicines} \& \measure{at once} \& \timexDate{today}.}{
\discontRel{2}{4}
}
\end{flushleft}


\paragraph{Ambiguity} Natural language is often ambiguous, and more so for colloquial language taken out of a conversation online. 
In many cases, users are no explicitly stating, e.g., a specific test they did.

\begin{flushleft}
\test{EKG} \measure{perfect}, \test{tri's} were \measure{perfect}, \test{blood sugar} \measure{perfect}, etc.
\end{flushleft}
In the above example, it is not entirely clear whether \textit{blood sugar} is a \texttt{function} or a \texttt{test}. 
Following up on this, is \textit{perfect} an \texttt{opinion} or a \texttt{measure}?
We decide to treat implicit mentions as if they were explicit, therefore, in this example, we annotate \textit{EKG}, \textit{tri}, as well as \textit{blood sugar} as \texttt{test}, and \textit{perfect} as \texttt{measure}, as if the person would have given exact test results an reported that they did a ``blood sugar test'', e.g., a A1C test.
    

\paragraph{Specific manifestations of the same function} 

\begin{flushleft}
\drug{Lipidor} \opPos{did the job} on my \function{cholesterol} both \emph{LDL} and \emph{HDL}.
\end{flushleft}

$\rightarrow$ Should \textit{LDL} and \textit{HDL} be annotated as \texttt{function}?


\subsection{French}

\begin{description}
    \item[Unclear metaphors] (\ldots), des \disorder{boutons} \anatomy{de la tête au pied}, (\ldots)\\
    $\rightarrow$ What exactly does the person mean?
    
    \item[Function or not?] Suite à ma deuxième \other{grossesse} j'ai souhaité passé sur un traitement à faire à domicile, (\ldots) \translation{After my second \other{pregnancy} I wanted to go ahead with home care, (\ldots)}\\
    Depending on the context, \textit{grossesse} (pregnancy) might be a function. UMLS is not clear about that.

\end{description}

\subsection{German}

\begin{description}
    \item[Descriptive Language] ``Danach bin ich erstmal immer im Tal.''
\end{description}

\subsection{Japanese}

\subsubsection{Body weight changes with weight values}

\begin{itemize}
    \item \drawRelations{%
        \timexDur{1日で} \& \disorder{体重} \& 2キロ \& \disorder{減った}
        }{
        \discontRel{1}{4}
        }
        \translation{\disorder{my weight} \disorder{dropped} 2 kilograms \timexDur{in one day}.}
        
        Here, ``2キロ (two kilograms)'' is inserted between ``体重 (my weight)'' and ``減った (dropped)'' in the original Japanese sentence, which forced the annotation to be \textit{discontinuous}.

    \item \drawRelations{%
        \other{手術}前にくらべ \& \disorder{体重} \& が8キロ\& \disorder{増加} \& しました}{
        \discontRel{2}{4}
        }
        \translation{\disorder{My weight} has \disorder{increased} by 8 kg compared to before the \other{surgery}.}

        ``8キロ (8 kg)'' also splits the ``体重が増加 (my weight has increased'' in this case, resulting in a discontinuous annotation
\end{itemize}



% Potential additional entities: health-improvement, lifestyle (change)

\end{document}
